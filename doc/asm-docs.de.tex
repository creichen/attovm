\documentclass[11pt,a4paper]{article} 
\usepackage[margin=2cm]{geometry}
\usepackage[ngerman]{babel}
\usepackage[utf8]{inputenc}
\usepackage{hyperref}
%% \documentclass{article}
\usepackage{color}
\usepackage{listings}
\usepackage{2opm}
\usepackage{tikz}
\usepackage{pgf}
\definecolor{dblue}{rgb}{0,0,0.5}

%% \newcommand{\textdbf}[1]{\texttt{\textcolor{dblue}{\textbf{#1}}}}
 
\lstset{ %
  language=[2opm]Assembler,       % the language of the code
  basicstyle=\small\tt,       % the size of the fonts that are used for the code
%  numbers=left,                   % where to put the line-numbers
  numberstyle=\small\color{grey},  % the style that is used for the line-numbers
  stepnumber=1,                   % the step between two line-numbers. If it's 1, each line 
                                  % will be numbered
  numbersep=5pt,                  % how far the line-numbers are from the code
  backgroundcolor=\color{white},  % choose the background color. You must add \usepackage{color}
  showspaces=false,               % show spaces adding particular underscores
  showstringspaces=false,         % underline spaces within strings
  showtabs=false,                 % show tabs within strings adding particular underscores
%  frame=single,                   % adds a frame around the code
  rulecolor=\color{black},        % if not set, the frame-color may be changed on line-breaks within not-black text (e.g. commens (green here))
  tabsize=4,                      % sets default tabsize to 2 spaces
  captionpos=b,                   % sets the caption-position to bottom
  breaklines=true,                % sets automatic line breaking
  breakatwhitespace=false,        % sets if automatic breaks should only happen at whitespace
  title=\lstname,                 % show the filename of files included with \lstinputlisting;
                                  % also try caption instead of title
  keywordstyle=\textdbf,           % keyword style
  commentstyle=\color{dgreen},       % comment style
  stringstyle=\color{mauve},         % string literal style
  escapeinside={\%*}{*)},            % if you want to add a comment within your code
  morekeywords={*,eingabe,ausgabe,...},               % if you want to add more keywords to the set
  belowskip=-1em
}


\tikzstyle{ptr} = [thick, > = stealth]


\newcommand{\textdbf}[1]{\texttt{\textcolor{dblue}{\textbf{#1}}}}
 
\newcommand{\insn}[1]{\texttt{\textbf{\textcolor{dblue}{#1}}}}
\title{2OPM-Assembler: Befehlsliste\\ \normalsize v0.1.1}

\begin{document}
\maketitle

2OPM (Zwei-Operanden-Pseudo-MIPS) ist eine Assemblersprache, die in eine Maschinensprache übersetzt werden kann,
die auf x86-64 CPUs direkt ausführbar ist.   Die Sprache wird also üblicherweise nicht
interpretiert sondern als nativer Maschinencode von der CPU direkt ausgeführt.

Die Sprache ist stark an MIPS-Assembler angelehnt.  Die wesentlichsten Unterschiede sind:
\begin{itemize}
\item Weniger Register
\item 64-Bit-Register
\item Zwei-Operanden-Befehle für die meisten arithmetischen und Bit-Befehle: Statt
\[
  \texttt{\textdbf{add} \$t0, \$t0, \$t1}
\]
    schreiben wir nur
\[
    \texttt{\textdbf{add} \$t0, \$t1}
\]
    Das linke Eingaberegister wird von arithmetischen und logischen Operationen also überschrieben.
\item Andere Aufrufkonventionen:
  \begin{itemize}
  \item 6 Argumentregister \texttt{\$a0}\ldots\texttt{\$a5}
  \item Nur eine Rückgaberegister \texttt{\$v0}
  \item Kein \texttt{\$ra}-Register: \textdbf{jal} und \textdbf{jalr} legen die Rücksprungadressen auf dem Stapel ab und ziehen
    automatisch 8 von \texttt{\$sp} ab
  \end{itemize}
  \item Statt \textdbf{jr} verwenden wir \textdbf{jreturn}, der die
    Rücksprungadresse vom Stapel bezieht und das Stapelregister
    automatisch um 8 erhöht.
\end{itemize}

2OPM kann entweder separat betrieben werden oder als Zielsprache für
einen Übersetzer (z.B. \texttt{AttoVM}) dienen.  Dieses Dokument
beschreibt 2OPM für sich selbst genommen; die Beschreibung von AttoVM
findet sich in einem separaten Dokument.

\section{Register}

  Der vom Übersetzer verwendete 2OMP-Code ist an MIPS-Code angelehnt.  Da die x86-64-Architektur aber
fundamental anders aufgebaut ist, ist die Maschinencoderepräsentierung von Assemblerbefehlen oft deutlich anders.
%% Zudem verfügt die x86-64-Architektur über weniger Register:

%% \begin{tabular}{|l|l|l|l|l|l|}
%% \hline
%% \textbf{Rückgabe} & \textbf{Temporär} & \textbf{Gesichert} & \textbf{Parameter} & \textbf{Spezial} & \textbf{Programmzähler}\\
%% \hline
%% \hline
%% \texttt{\$v0} & \texttt{\$t0} & \texttt{\$s0} & \texttt{\$a0} & \texttt{\$gp} & \texttt{\$pc} \\
%%               & \texttt{\$t1} & \texttt{\$s1} & \texttt{\$a1} & \texttt{\$sp} &               \\
%%               &               & \texttt{\$s2} & \texttt{\$a2} & \texttt{\$fp} &               \\
%%               &               & \texttt{\$s3} & \texttt{\$a3} &               &               \\
%%               &               &               & \texttt{\$a4} &               &               \\
%%               &               &               & \texttt{\$a5} &               &               \\
%% \hline


2OPM-Assembler verwendet die folgenden Allzweck-Register:


  \begin{tabular}{|p{6cm}|p{6cm}|}
    \hline
    \textbf{Name} & \textbf{Zweck} \\
    \hline
    \hline
    \texttt{\$v0} & Rückgabewert (return \textbf{v}alue)\\
    \texttt{\$a0}, \texttt{\$a1}, \texttt{\$a2}, \texttt{\$a3}, \texttt{\$a4}, \texttt{\$a5} & \textbf{A}rgumentregister\\
    \texttt{\$s0}, \texttt{\$s1}, \texttt{\$s2}, \texttt{\$s3} & Vom Aufrufer ge\textbf{s}icherte Register \\
    \texttt{\$t0}, \texttt{\$t1} & \textbf{t}emporäre Register \\
    \texttt{\$sp} & Stapelzeiger (\textbf{s}tack \textbf{p}ointer)\\
    \texttt{\$fp} & Rahmenzeiger (\textbf{f}rame \textbf{p}ointer)\\
    \texttt{\$gp} & Globaler Zeiger (\textbf{g}lobal \textbf{p}ointer) in das statische Datensegment\\
    \hline
  \end{tabular}


Das Register \texttt{\$pc} ist der Programmzähler.  Dieses Register kann nicht direkt gelesen oder beschrieben werden, sondern nur indirekt über Sprungbefehle.

Wenn ein Assemblerprogramm geladen wird, stellt der Lader sicher, daß die folgenden Register auf `vernünftige' Werte gesetzt werden,
bevor das Programm gestartet wird:
\begin{itemize}
  \item \texttt{\$sp} ist eine nutzbare Stapeladresse, und \texttt{\$sp} modulo 16 ist 0 (wie nach einem normalen Subroutinen-Aufruf, siehe Abschnitt~\ref{sec:calling}).
  \item \texttt{\$gp} zeigt in den statischen Speicher / \texttt{.data}/\texttt{.bss}.
\end{itemize}
Der Lader stellt zudem sicher, daß eine nutzbare Rücksprungadresse auf dem Stapel liegt, so daß das geladene Programm mit
sich mit \texttt{jreturn} beenden kann.

\section{Speicher}
Der Lader stellt zwei gesonderte Speicherregionen zur Verfügung, die das geladene Assemblerprogramm verwenden kann:
\begin{itemize}
\item Eine Code-Region (die implizit vom Programmzähler verwendet wird)
\item Eine statische Daten-Region (die von \texttt{\$gp} referenziert wird)
\end{itemize}
Das Assemblerprogramm verwendet den Stapel des Laders.

\section{Befehle}

Assemblerprogramme bestehen aus Folgen von Assembler-Instruktionen.  Die möglichen Instruktionen sind unten aufgeführt, mit kurzen Erklärungen.
Die meisten Befehle nehmen einen oder mehrere Parameter.  Wir unterscheiden zwischen den folgenden Parameterformen:
\begin{itemize}
  \item \texttt{addr}: Eine Speicheradresse; bei 2OPM-Programmierung `von Hand' sind dies üblicherweise (Sprung)marken (labels)
  \item \texttt{u8}: Eine nicht vorzeichenbehaftete 8-bit-Zahl
  \item \texttt{u32}: Eine nicht vorzeichenbehaftete 32-bit-Zahl
  \item \texttt{s32}: Eine vorzeichenbehaftete 32-bit-Zahl
  \item \texttt{u64}: Eine nicht vorzeichenbehaftete 64-bit-Zahl
  \item \texttt{\$r0}, \texttt{\$r1}, \texttt{\$r2}: Ein beliebiges Allzweck-Register
\end{itemize}

\pagenumbering{gobble}
\begin{tabular}{llp{8cm}}
\small
\textcolor{dblue}{\textbf{\texttt{move}}}&      \texttt{\$r0}, \texttt{\$r1}&   $\texttt{\$r}_{0}$ := $\texttt{\$r}_{1}$\\
\textcolor{dblue}{\textbf{\texttt{li}}}&        \texttt{\$r0}, s64&     $\texttt{\$r}_{0}$ := \texttt{s64}\\
\textcolor{dblue}{\textbf{\texttt{add}}}&       \texttt{\$r0}, \texttt{\$r1}&   $\texttt{\$r}_{0}$ := $\texttt{\$r}_{0}$ + $\texttt{\$r}_{1}$\\
\textcolor{dblue}{\textbf{\texttt{addi}}}&      \texttt{\$r0}, u32&     $\texttt{\$r}_{0}$ := $\texttt{\$r}_{0}$ + \texttt{u32}\\
\textcolor{dblue}{\textbf{\texttt{sub}}}&       \texttt{\$r0}, \texttt{\$r1}&   $\texttt{\$r}_{0}$ := $\texttt{\$r}_{0}$ - $\texttt{\$r}_{1}$\\
\textcolor{dblue}{\textbf{\texttt{subi}}}&      \texttt{\$r0}, u32&     $\texttt{\$r}_{0}$ := $\texttt{\$r}_{0}$ $-$ \texttt{u32}\\
\textcolor{dblue}{\textbf{\texttt{mul}}}&       \texttt{\$r0}, \texttt{\$r1}&   $\texttt{\$r}_{0}$ := $\texttt{\$r}_{0}$ * $\texttt{\$r}_{1}$\\
\textcolor{dblue}{\textbf{\texttt{div\_a2v0}}}& \texttt{\$r0}&  \texttt{\$v0} := \texttt{\$a2}:\texttt{\$v0} / $\texttt{\$r}_{0}$, \texttt{\$a2} := Rest\\
\textcolor{dblue}{\textbf{\texttt{not}}}&       \texttt{\$r0}, \texttt{\$r1}&   if $\texttt{\$r}_{1}$ = 0 then $\texttt{\$r}_{1}$ := 1 else $\texttt{\$r}_{1}$ := 0\\
\textcolor{dblue}{\textbf{\texttt{and}}}&       \texttt{\$r0}, \texttt{\$r1}&   $\texttt{\$r}_{0}$ := $\texttt{\$r}_{0}$ Bitwise-Und $\texttt{\$r}_{1}$\\
\textcolor{dblue}{\textbf{\texttt{andi}}}&      \texttt{\$r0}, u32&     $\texttt{\$r}_{0}$ := $\texttt{\$r}_{0}$ Bitwise-Und \texttt{u32}\\
\textcolor{dblue}{\textbf{\texttt{or}}}&        \texttt{\$r0}, \texttt{\$r1}&   $\texttt{\$r}_{0}$ := $\texttt{\$r}_{0}$ Bitwise-Oder $\texttt{\$r}_{1}$\\
\textcolor{dblue}{\textbf{\texttt{ori}}}&       \texttt{\$r0}, u32&     $\texttt{\$r}_{0}$ := $\texttt{\$r}_{0}$ Bitwise-Oder \texttt{u32}\\
\textcolor{dblue}{\textbf{\texttt{xor}}}&       \texttt{\$r0}, \texttt{\$r1}&   $\texttt{\$r}_{0}$ := $\texttt{\$r}_{0}$ Bitwise-Exklusiv-Oder $\texttt{\$r}_{1}$\\
\textcolor{dblue}{\textbf{\texttt{xori}}}&      \texttt{\$r0}, u32&     $\texttt{\$r}_{0}$ := $\texttt{\$r}_{0}$ Bitwise-Exklusiv-Oder \texttt{u32}\\
\textcolor{dblue}{\textbf{\texttt{sll}}}&       \texttt{\$r0}, \texttt{\$r1}&   $\texttt{\$r}_{0}$ := $\texttt{\$r}_{0}$ ${<}{<}$ $\texttt{\$r}_{1}$[0:7]\\
\textcolor{dblue}{\textbf{\texttt{slli}}}&      \texttt{\$r0}, \texttt{\$r1}, u8&       $\texttt{\$r}_{0}$ := $\texttt{\$r}_{0}$ um \texttt{u8} Bits nach links\\
\textcolor{dblue}{\textbf{\texttt{srl}}}&       \texttt{\$r0}, \texttt{\$r1}&   $\texttt{\$r}_{0}$ := $\texttt{\$r}_{0}$ ${>}{>}$ $\texttt{\$r}_{1}$[0:7]\\
\textcolor{dblue}{\textbf{\texttt{srli}}}&      \texttt{\$r0}, \texttt{\$r1}, u8&       $\texttt{\$r}_{0}$ := $\texttt{\$r}_{0}$ um \texttt{u8} Bits nach rechts\\
\textcolor{dblue}{\textbf{\texttt{sra}}}&       \texttt{\$r0}, \texttt{\$r1}&   $\texttt{\$r}_{0}$ := $\texttt{\$r}_{0}$ ${>}{>}$ $\texttt{\$r}_{1}$[0:7], Vorzeichenerweitert\\
\textcolor{dblue}{\textbf{\texttt{srai}}}&      \texttt{\$r0}, u8&      $\texttt{\$r}_{0}$ := $\texttt{\$r}_{0}$ um \texttt{u8} Bits nach rechts, Vorzeichenerweitert\\
\textcolor{dblue}{\textbf{\texttt{slt}}}&       \texttt{\$r0}, \texttt{\$r1}, \texttt{\$r2}&    wenn $\texttt{\$r}_{1}$ < $\texttt{\$r}_{2}$ dann $\texttt{\$r}_{1}$ := 1 sonst $\texttt{\$r}_{1}$ := 0\\
\textcolor{dblue}{\textbf{\texttt{sle}}}&       \texttt{\$r0}, \texttt{\$r1}, \texttt{\$r2}&    wenn $\texttt{\$r}_{1}$ $\le$ $\texttt{\$r}_{2}$ dann $\texttt{\$r}_{1}$ := 1 sonst $\texttt{\$r}_{1}$ := 0\\
\textcolor{dblue}{\textbf{\texttt{seq}}}&       \texttt{\$r0}, \texttt{\$r1}, \texttt{\$r2}&    wenn $\texttt{\$r}_{1}$ = $\texttt{\$r}_{2}$ dann $\texttt{\$r}_{1}$ := 1 sonst $\texttt{\$r}_{1}$ := 0\\
\textcolor{dblue}{\textbf{\texttt{sne}}}&       \texttt{\$r0}, \texttt{\$r1}, \texttt{\$r2}&    wenn $\texttt{\$r}_{1}$ $\ne$ $\texttt{\$r}_{2}$ dann $\texttt{\$r}_{1}$ := 1 sonst $\texttt{\$r}_{1}$ := 0\\
\textcolor{dblue}{\textbf{\texttt{bgt}}}&       \texttt{\$r0}, \texttt{\$r1}, addr&     wenn $\texttt{\$r}_{0}$ $>$ $\texttt{\$r}_{1}$, dann springe nach \texttt{addr}\\
\textcolor{dblue}{\textbf{\texttt{bge}}}&       \texttt{\$r0}, \texttt{\$r1}, addr&     wenn $\texttt{\$r}_{0}$ $\ge$ $\texttt{\$r}_{1}$, dann springe nach \texttt{addr}\\
\textcolor{dblue}{\textbf{\texttt{blt}}}&       \texttt{\$r0}, \texttt{\$r1}, addr&     wenn $\texttt{\$r}_{0}$ $<$ $\texttt{\$r}_{1}$, dann springe nach \texttt{addr}\\
\textcolor{dblue}{\textbf{\texttt{ble}}}&       \texttt{\$r0}, \texttt{\$r1}, addr&     wenn $\texttt{\$r}_{0}$ $\le$ $\texttt{\$r}_{1}$, dann springe nach \texttt{addr}\\
\textcolor{dblue}{\textbf{\texttt{beq}}}&       \texttt{\$r0}, \texttt{\$r1}, addr&     wenn $\texttt{\$r}_{0}$ = $\texttt{\$r}_{1}$, dann springe nach \texttt{addr}\\
\textcolor{dblue}{\textbf{\texttt{bne}}}&       \texttt{\$r0}, \texttt{\$r1}, addr&     wenn $\texttt{\$r}_{0}$ $\ne$ $\texttt{\$r}_{1}$, dann springe nach \texttt{addr}\\
\textcolor{dblue}{\textbf{\texttt{bgtz}}}&      \texttt{\$r0}, addr&    wenn $\texttt{\$r}_{0}$ $>$ 0, dann springe nach \texttt{addr}\\
\textcolor{dblue}{\textbf{\texttt{bgez}}}&      \texttt{\$r0}, addr&    wenn $\texttt{\$r}_{0}$ $\ge$ 0, dann springe nach \texttt{addr}\\
\textcolor{dblue}{\textbf{\texttt{bltz}}}&      \texttt{\$r0}, addr&    wenn $\texttt{\$r}_{0}$ $<$ 0, dann springe nach \texttt{addr}\\
\textcolor{dblue}{\textbf{\texttt{blez}}}&      \texttt{\$r0}, addr&    wenn $\texttt{\$r}_{0}$ $\le$ 0, dann springe nach \texttt{addr}\\
\textcolor{dblue}{\textbf{\texttt{bnez}}}&      \texttt{\$r0}, addr&    wenn $\texttt{\$r}_{0}$ $\ne$ 0, dann springe nach \texttt{addr}\\
\textcolor{dblue}{\textbf{\texttt{beqz}}}&      \texttt{\$r0}, addr&    wenn $\texttt{\$r}_{0}$ = 0, dann springe nach \texttt{addr}\\
\textcolor{dblue}{\textbf{\texttt{j}}}& addr&   springe nach \texttt{addr}\\
\textcolor{dblue}{\textbf{\texttt{jr}}}&        \texttt{\$r0}&  springe nach \texttt{\$r0}\\
\textcolor{dblue}{\textbf{\texttt{jal}}}&       addr&   Nächste Adresse auf den Stapel, springe nach \texttt{addr}\\
\textcolor{dblue}{\textbf{\texttt{jalr}}}&      \texttt{\$r0}&  Nächste Adresse auf den Stapel, springe nach $\texttt{\$r}_{0}$\\
\textcolor{dblue}{\textbf{\texttt{jreturn}}}&   &       springe nach mem64[\texttt{\$sp}]; \texttt{\$sp} := \texttt{\$sp} + 8\\
\textcolor{dblue}{\textbf{\texttt{sb}}}&        \texttt{\$r0}, s32, \texttt{\$r1}&      mem8[$\texttt{\$r}_{1}$ + \texttt{s32}] := $\texttt{\$r}_{0}$[7:0]\\
\textcolor{dblue}{\textbf{\texttt{lb}}}&        \texttt{\$r0}, s32, \texttt{\$r1}&      mem8[$\texttt{\$r}_{1}$ + \texttt{s32}] := $\texttt{\$r}_{0}$[7:0]\\
\textcolor{dblue}{\textbf{\texttt{sd}}}&        \texttt{\$r0}, s32, \texttt{\$r1}&      mem64[$\texttt{\$r}_{1}$ + \texttt{s32}] := $\texttt{\$r}_{0}$\\
\textcolor{dblue}{\textbf{\texttt{ld}}}&        \texttt{\$r0}, s32, \texttt{\$r1}&      $\texttt{\$r}_{0}$ := mem64[$\texttt{\$r}_{1}$ + \texttt{s32}]\\
\textcolor{dblue}{\textbf{\texttt{syscall}}}&   &       Systemaufruf\\
\textcolor{dblue}{\textbf{\texttt{push}}}&      \texttt{\$r0}&  \texttt{\$sp} := \texttt{\$sp} - 8; mem64[\texttt{\$sp}] = $\texttt{\$r}_{0}$\\
\textcolor{dblue}{\textbf{\texttt{pop}}}&       \texttt{\$r0}&  $\texttt{\$r}_{0}$ = mem64[\texttt{\$sp}]; \texttt{\$sp} := \texttt{\$sp} + 8\\
\end{tabular}
\newpage
\pagenumbering{arabic}

\section{Aufrufkonventionen}\label{sec:calling}

2OPM folgt der x86-64/Linux ABI (Application Binary Interface), übersetzt in die 2OPM-Registernamen.  Dies erlaubt 2OPM-Code,
direkt C-Funktionen aus dem Assembler heraus aufzurufen, da diese der gleichen ABI folgen.

\subsection{Vorbereitung vor Aufruf}
  \begin{itemize}
    \item Die ersten sechs Parameter werden in \texttt{\$a0}\ldots\texttt{\$a5} abgelegt
    \item Zusätzliche Parameter werden auf dem Stapel abgelegt:
      \begin{itemize}
        \item Argument 6 (7th): im Speicher bei \texttt{\$sp}
        \item Argument 7 (8th): im Speicher bei $\texttt{\$sp} + 8$\\
          \ldots
      \end{itemize}
    \item $\texttt{\$sp} + 8$ ist \emph{$128$-bit-ausgerichtet} (teilbar durch $16$)
  \end{itemize}

\subsection{Betreten einer Subroutine}
    \begin{itemize}
    \item $\texttt{\$sp}$ ist \emph{$128$-Bit-ausgerichtet}
    \item Speicherinhalt des Speichers, auf den \texttt{\$sp} zeigt, ist die die Rücksprungadresse
    \end{itemize}

    Der \textdbf{jal}-Befehl (ebenso \textdbf{jalr}) stellt diese Eigenschaften implizit sicher, wenn
    die korrekten Vorbereitungen durchgeführt wurden.

\subsection{Während des Subroutinenaufrufs}
    \begin{itemize}
    \item $\texttt{\$fp} + 8$ ist \emph{$128$-bit-ausgerichtet}
    \item Die Funktion hat einen \emph{Aktivierungseintrag}:

      \begin{tikzpicture}
        \node[dashed, draw, minimum height=0.6cm, minimum width=4cm] at (0, 1.2) {$\cdots$};
        \node[draw, minimum height=0.6cm, minimum width=4cm] at (0, 0.6) {Argument \#7};
        \node[draw, minimum height=0.6cm, minimum width=4cm] at (0, 0.0) {Argument \#6};
        \node[draw, minimum height=0.6cm, minimum width=4cm] at (0, -0.6) {Rücksprungadresse};
        \node[draw, minimum height=0.6cm, minimum width=4cm] at (0, -1.2) (FPOLDBASE) {\texttt{\$fp} des Aufrufers};
        \node[draw, minimum height=1.8cm, minimum width=4cm] at (0, -2.4) (LOCALS) {Lokale Variablen};

        \node at (FPOLDBASE.south west) (FPOLD) {};
        \node[left of=FPOLD, node distance=4cm] (FP) {\texttt{\$fp}};
        \node at (LOCALS.south west) (LOCALSEND) {};
        \node[left of=LOCALSEND, node distance=4cm] (SP) {\texttt{\$sp}};
        \draw[ptr, ->] (FP) -- (FPOLD);
        \draw[ptr, ->] (SP) -- (LOCALSEND);
      \end{tikzpicture}

      Die Daten werden also wie folgt abgelegt:
      \begin{itemize}
        \item Argument \#7 (8tes): in \texttt{\$fp} + 24 (etc.)
        \item Argument \#6 (7tes): in \texttt{\$fp} + 16 gespeichert
        \item Rücksprungadresse in \texttt{\$fp} + 8 gespeichert
        \item \texttt{\$fp} des Aufrufers in \texttt{\$fp} gespeichert
        \item Lokale Variablen beginnen bei \texttt{\$fp} - 8
      \end{itemize}
    \end{itemize}

    Die ABI erlaubt es auch, \texttt{\$fp} nicht zu speichern\footnote{Diese Optimierung wird von AttoVM 0.4 nicht verwendet.}.
    \texttt{\$fp} zeigt in diesem Fall direkt auf die erste lokale Variabe.

\subsection{Rücksprung von einer Subroutine}
  \begin{itemize}
    \item Die folgenden Register müssen beim Rücksprung die gleichen Inhalte wie beim Betreten der Subroutine haben:
      \begin{itemize}
        \item \texttt{\$sp}
        \item \texttt{\$fp}
        \item \texttt{\$gp}
        \item \texttt{\$s0}--\texttt{\$s3}
      \end{itemize}
    \item Alle anderen Register \emph{dürfen} modifiziert werden.
    \item Das Register \texttt{\$v0} beinhaltet den Rückgabewert, falls es einen solchen gibt.
  \end{itemize}

\section{Kommandozeilen-Assembler}

Das  2OPM-Assemblerwerkzeug (\texttt{asm}) kann Assemblerdateien 
(mit dem Dateisuffix \texttt{.s}, per Konvention) laden, binden, und ausführen.  Es bietet zudem einen Debugger an,
der durch
Code schreiten, Register ausgeben, und Stapelinhalte und statischen Speicher ausgeben kann.
Der Assembler unterstützt zudem einen Breakpoint.

Eine Beschreibung des Assemblers findet sich in der englischsprachigen Dokumentation.


%% \subsection{Installing the Assembler}
%% Download the assembler from the specified location.  If you unpack it
%% on a UNIX command line and run `\texttt{make}', it should compile
%% and link a program `\texttt{asm}'.  This program is a stand-alone executable
%% and can be run from any location.

%% \subsection{Using the Assembler}
%% To start the assembler, write a small assembly program, store it in the file \texttt{myprogram.s} in the same
%% directory that contains your \texttt{asm} executable, and run
%% \[
%% \texttt{./asm myprogram.s}
%% \]

%% on the command shell in that directory (see below for some sample programs).

%% \subsubsection{Using the Debugger}
%% To activate the debugger, start the assembler with the command line option \texttt{-d}.
%% The debugger has a built-in help facility that can be activated by writing \texttt{help} as
%% soon as the debugger command prompt appears.

%% \subsection{Assembler Programs}

%% The assembler takes four kinds of input:
%% \begin{itemize}
%%   \item \emph{assembly instructions},
%%   \item \emph{labels},
%%   \item \emph{directives}, which control the meaning of subsequent input, and
%%   \item \emph{data}.
%% \end{itemize}

%% A functional program must provide at least the first three; providing
%% data is optional.

%% As an example, consider this program:

%% \begin{lstlisting}
%% .text
%% main:
%%         push $t0        ; align stack for subroutine call
%%         li   $a0, 42
%%         jal  print_int  ; call built-in function to print
%%         pop  $t0
%%         jreturn
%% \end{lstlisting}
%% (Note the comment syntax.)

%% This program consists of five assembly instructions, one of which
%% calls a built-in function (see below).  The first two lines, however,
%% are not assembly instructions.  Here, \texttt{.text} indicates that
%% the following output should go into the text segment.  The assembler
%% will permit assembly instructions if and only if the text segment has
%% been selected.  The label \texttt{main} marks the main entry point.
%% Each executable assembly program MUST define a \texttt{main} entry
%% point.  Any further labels are optional.

%% \subsection{Directives}

%% 2OPM supports five directives.  The two most important ones are \texttt{.text} and \texttt{.data}.

%% \paragraph{\texttt{.text}} indicates that any following information goes into the text segment, 
%% i.e., is intended for execution.  The following information must be assembly instructions
%% and may include label definitions and label references (for suitable instructions).

%% \paragraph{\texttt{.data}} indicates  that any following information is pure data.  No 
%% assembly instructions are permitted (this is for simplicity; in principle, the
%% computer could represent assembly instructions in static memory).  The data section 
%% permits label definitions, and
%% freely
%% mixes all forms of data; however, introducing data requires selecting a \emph{data mode}.

%% \subsection{Data modes}

%% The following data modes are permitted:

%% \paragraph{\texttt{.byte}} allows inserting single bytes, separated by commas.

%% \paragraph{\texttt{.word}} allows inserting 64-bit words, separated by commas.  This section
%% also permits label references: the labels' memory address are included verbatim.

%% \paragraph{\texttt{.asciiz}} allows ASCII character strings.  All strings are automatically
%% zero-terminated.

%% As an example for using the data segment, consider the following:

%% \begin{lstlisting}
%% .text
%% main:
%%         push $fp      ; align stack
%%         move $fp, $sp

%%         la   $a0, hello
%%         jal  print_string  ; print out
%%         ld   $a0, number($gp)
%%         jal  print_int
%%         la   $t0, more_numbers
%%         ld   $v0, 0($t0)
%%         ld   $a0, 8($t0)
%%         add  $a0, $v0
%%         jal  print_int     ; print sum

%%         pop  $fp
%%         jreturn
%% .data
%% hello:
%% .asciiz "Hello, World!"
%% .word
%% number:
%% 	23
%% more_numbers:
%% 	3,4
%% \end{lstlisting}

%% \subsection{Labels}

%% Labels are defined by writing the label's name, followed by a colon, as in \texttt{label:}.
%% References to labels are written by omitting the colon.  Each label may be defined only once,
%% but may be referenced arbitrarily many times.

%% Label references are permitted in \texttt{.data} sections in \texttt{.word} mode, and in
%% assembly instructions such as \texttt{\textcolor{dblue}{j}} or \texttt{\textcolor{dblue}{blt}}.  The
%% assembler automatically figures out whether the references are relative or absolute and relocates suitably.

%% To load a memory address of any label directly, the assembler provides a pseudo-instruction:

%% \[
%% \texttt{\textcolor{dblue}{la} \texttt{\$r}, \texttt{\textit{addr}}}
%% \]

%% This instruction loads the absolute memory address of the specified label into register \texttt{\$r},
%% no matter whether the address is in text or (static) data memory.


%% \subsubsection{Built-in Operations}
%% The 2OPM runtime comes with a small number of built-in subroutines to facilitate
%% input and output.  Each of them can be called using
%% \texttt{\textcolor{dblue}{jal}}:

%% \begin{tabular}{ll}
%% \texttt{print\_int} & Print parameter as signed integer \\
%% \texttt{print\_string} & Print parameter as null-terminated ascii string \\
%% \texttt{read\_int} & Read and return a single 64 bit integer \\
%% \texttt{exit} & Stop the program \\
%% \end{tabular}

%% These functions use system calls (cf. the
%% \texttt{\textcolor{dblue}{syscall}} operation) to achieve special
%% effects that require interaction with the operating system; they are
%% abstracted for convenience.


%% For example, the following program will read two numbers, add them,
%% and print the resultant output:

%% \begin{lstlisting}
%% .text
%% main:
%%         push $fp
%%         move $fp, $sp ; align stack
%%         jal  read_int
%%         move $s0, $v0
%%         jal  read_int
%%         move $a0, $s0
%%         add  $a0, $v0
%%         jal  print_int
%%         pop  $fp
%%         jreturn
%% \end{lstlisting}


%% \section{Implementation Notes}

%% Most 2OPM instructions correspond directly to x86-64 instructions.
%% However, some of the instruction choices are not optimal: for example,
%% 2OPM always uses 64 bit load operations, even if the number loaded can
%% be represented in 32 or fewer bits.  Some other operations are
%% emulated: x86-64 only permits bit shifting by register \texttt{cl}
%% (\texttt{\$a2}[7:0]), so the 2OPM implementation of this instruction
%% introduces additional register-swap operations, if needed.  This may
%% result in less-than-optimal performance.

\end{document}
