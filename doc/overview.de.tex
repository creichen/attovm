\documentclass[11pt,a4paper]{article} 
\usepackage[margin=2cm]{geometry}
\usepackage[ngerman]{babel}
\usepackage[utf8]{inputenc}
\usepackage{hyperref}
\usepackage{tikz}
\usepackage{pgf}
\usepackage{color}
\usepackage{moresize}
\usepackage{xcolor}
\usepackage{listings}
\usepackage{wasysym}
%\usepackage{mips}
\usepackage{pgfpages}
\renewcommand{\theenumi}{\alph{enumi}}
\renewcommand{\theenumii}{\roman{enumii}}
\usetikzlibrary{decorations.pathreplacing, backgrounds, trees, patterns, arrows, shapes, decorations.markings, calc, positioning, fit, matrix, decorations.pathmorphing}
\renewcommand*\ttdefault{txtt}


\newcommand{\mybackslash}{\backslash{}{}{}{}{}{}{}}
\newenvironment{slisting}{%
        \begin{tt}%
        \begin{tabular}{l}%
        }
        {%
        \end{tabular}%
        \end{tt}%
        }

\definecolor{dcyan}{rgb}{0,0.4,0.4}

\newcommand{\Cinclude}[1]{\textcolor{dgreen}{\#include $<$#1$>$}}
\newcommand{\Cty}[1]{\textcolor{dblue}{\texttt{#1}}}
\newcommand{\Ccom}[1]{\textcolor{dgreen}{\texttt{#1}}}
\newcommand{\Cpp}[1]{\textcolor{dcyan}{\texttt{#1}}}
\newcommand{\Ckw}[1]{\textbf{\texttt{#1}}}


\newcommand{\coloneq}{\mathrel{\mathop:}=}
\newcommand{\bincode}[1]{\texttt{\textcolor{blue}{#1}}}
\newcommand{\black}[1]{{\color{black}#1}}
\newcommand{\bincodeB}[1]{\texttt{\textbf{\textcolor{black}{#1}}}}

\newcommand{\ONE}{\texttt{1}}
\newcommand{\ZERO}{\texttt{0}}
\newcommand{\NO}{\node {\ONE};}
\newcommand{\NZ}{\node {\ZERO};}

%% \setbeameroption{show notes}
%% \setbeameroption{show notes on second screen=right}

%\newcommand{\fquote}[1]{\frqq #1\flqq}
\newcommand{\fquote}[1]{\glqq #1\grqq}

\tikzstyle{dflowarrow} = [very thick, rounded corners, double distance=1pt, > = triangle 45 new, arrow head=3mm]
\tikzstyle{choicearrow} = [thick, arrow head=3mm]
\tikzstyle{editarrow} = [thick, decorate, decoration=snake, > = triangle 45 new, arrow head=3mm]

\definecolor{dkgreen}{rgb}{0,0.6,0}
\definecolor{dgreen}{rgb}{0,0.4,0}
\definecolor{dblue}{rgb}{0,0,0.5}
\definecolor{grey}{rgb}{0.5,0.5,0.5}
\definecolor{mauve}{rgb}{0.58,0,0.82}

\newcommand{\Prod}{::=}
\newcommand{\VB}{\ |\ }
\newcommand{\tuple}[1]{\ensuremath{\langle #1 \rangle}}
\newcommand{\nt}[1]{\ensuremath{\tuple{\hspace{-0.02cm}\nta{#1}\hspace{0.02cm}}}}
\newcommand{\sem}[1]{\ensuremath{\llbracket #1 \rrbracket}}
\newcommand{\nta}[1]{\ensuremath{\textit{#1}}}
\newcommand{\terminal}[1]{\textit{#1}}
\newcommand{\vterminal}[1]{\textsf{`}\texttt{#1}\textsf{'}}

%% \lstset{ %
%%   language=C       % the language of the code
%%   }
 
\lstset{ %
  language=C,       % the language of the code
  basicstyle=\small\tt,       % the size of the fonts that are used for the code
%  numbers=left,                   % where to put the line-numbers
  numberstyle=\small\color{grey},  % the style that is used for the line-numbers
  stepnumber=1,                   % the step between two line-numbers. If it's 1, each line 
                                  % will be numbered
  numbersep=5pt,                  % how far the line-numbers are from the code
  backgroundcolor=\color{white},  % choose the background color. You must add \usepackage{color}
  showspaces=false,               % show spaces adding particular underscores
  showstringspaces=false,         % underline spaces within strings
  showtabs=false,                 % show tabs within strings adding particular underscores
%  frame=single,                   % adds a frame around the code
  rulecolor=\color{black},        % if not set, the frame-color may be changed on line-breaks within not-black text (e.g. commens (green here))
  tabsize=4,                      % sets default tabsize to 2 spaces
  captionpos=b,                   % sets the caption-position to bottom
  breaklines=true,                % sets automatic line breaking
  breakatwhitespace=false,        % sets if automatic breaks should only happen at whitespace
  title=\lstname,                 % show the filename of files included with \lstinputlisting;
                                  % also try caption instead of title
  keywordstyle=\color{blue},          % keyword style
  commentstyle=\color{dkgreen},       % comment style
  stringstyle=\color{mauve},         % string literal style
  escapeinside={\%*}{*)},            % if you want to add a comment within your code
  morekeywords={*,eingabe,ausgabe,...},               % if you want to add more keywords to the set
  belowskip=-1em
}

\title{AttoL und AttoVM\\ \small{Sprachrevision 0.1}}
\begin{document}
\maketitle

AttoL ist eine kleine, überwiegend dynamisch getypte imperative
Programmiersprache mit statischem Scoping.  Die Sprache hat
eingeschränkte Unterstützung für objektorientierte Programmierung und ist
stark typisiert.

\section{Sprache}

Abbildung~\ref{fig:syntax} faßt die AttoL-Sprachsyntax zusammen.  Jedes
AttoL-Programm ist eine (ggf. leere) Folge von Anweisungen. 
Das Programm in Abbildung~\ref{fig:sieve} implementiert beispielsweise den
Sieb des Eratosthenes in AttoL.

\begin{figure}
\begin{slisting}
\Cty{int} size = 1000;\\
\Cty{int} max = 0;\\
\Cty{obj} sieve = [/size]; \Ccom{// allocate array}\\
\Cty{int} x = 2; \\
\Ckw{while} (x < size) \{ \\
\ \     \Ckw{if} (sieve[x] == \Ckw{NULL}) \{ \Ccom{// empty entry}\\
\ \ \ \        max := x; \\
\ \ \ \	\Cty{int} fill = x + x;\\
\ \ \ \         \Ckw{while} (fill < size) \{\\
\ \ \ \ \ \             sieve[fill] := x;\\
\ \ \ \ \ \             fill := fill + x;\\
\ \ \ \         \}\\
\ \     \}\\
\ \     x := x + 1;\\
\}\\
print(max);
\end{slisting}
\caption{Sieb des Eratosthenes in AttoL}\label{fig:sieve}
\end{figure}

\subsection{Statische Typen}\label{sec:types}
AttoL verwendet zwei statische Typen: \Cty{int}, eine vorzeichenbehaftete 64-Bit-Ganzzahl,
und \Cty{obj}, den Typ aller Sprachobjekte.
AttoL konvertiert zur Laufzeit implizit zwischen diesen beiden Typen (durch das \emph{Verpacken} und \emph{Entpacken} von Werten), so daß keine
statischen Typfehler auftreten können.
Der folgende Code illustriert dies:

\begin{slisting}
\Cty{obj} s = "{}string";\\
\Cty{int} i = 7;\\
s := 23;\ \ \  \Ccom{// Erlaubt; implizite Typkonvertierung (Verpacken)}\\
i := "{}foo"; \ \Ccom{// Laufzeit-Typfehler}\\
\end{slisting}

\Cty{int}-Variablen werden direkt als Zweierkomplementzahlen repräsentiert;
in \Cty{obj}-Variablen werden sie hingegen immer verpackt.

Der \Cty{obj}-Typ verfügt über den Sonderwert \Ckw{NULL},
der einen Wert beschreibt, der kein Objekt ist.

Variablen müssen bei Ihrer Deklaration nicht initialisiert werden,
aber es ist ein Fehler, eine Variable zu lesen, die nicht
initialisiert wurde.

AttoL unterstützt mehrere dynamische Typen:
\begin{itemize}
\item \Cty{int}: vorzeichenbehaftete 64-Bit-Ganzzahlen
\item \Cty{string}: Zeichenketten
\item \Cty{Array}: Arrays
\item Benutzerdefinierte Typen, die durch \Ckw{class}-Definitionen eingeführt wurden
\end{itemize}

Werte in allen diesen dynamische Typen können in \Cty{obj}-Variablen
gespeichert werden, aber nur \Cty{int}-Werte können in
\Cty{int}-Variablen gespeichert werden.

\subsection{Deklarationen und Definitionen}\label{sec:decl}
Zusätzlich zu Variablendeklarationen (Abschnitt~\ref{sec:types}) erlaubt AttoL zwei Arten von Definitionen:
\emph{Funktionsdefinitionen} und \emph{Klassendefinitionen}.  

Funktionsdefinitionen folgen der C-Syntax:

\begin{slisting}
\Cty{int} add(\Cty{int} a, \Cty{int} b) \{\\
\ \ \Ckw{return} a + b;\\
\}\\
\end{slisting}

Diese definiert eine Funktion \texttt{add} mit zwei Parametern, die die
\Cty{int}-parameter aufaddiert und das Ergebnis zurückliefert.
AttoL übergibt \Cty{int}-Parameter als Kopien des Wertes (`Wertparameter'), und
\Cty{obj}-Parameter als Zeiger (`Referenzparameter').

Klassendefinitionen sehen Funktionsdefinitionen ähnlich, beginnen aber mit dem Schlüsselwort \Ckw{class}:

\begin{slisting}
\Ckw{class} Box(\Cty{int} v) \{\\
\ \ \Cty{int} value = v;\\
\}\\
\end{slisting}

Dies definiert eine Klasse \Cty{Box} mit dem Feld \texttt{value}.  Sie können nun `\Cty{Box}'
wie eine Funktoin verwenden, um einen Wert dieses Typs zu erzeugen.  Die Parameter hinter dem
Klassennamen sind Parameter für den implizit definierten \emph{Konstruktor} der Klasse.
In diesem Fall sagen wir dem Konstruktor, daß er einen einzelnen Parameter, \texttt{v}, nehmen soll, und diesen
danach in dem Feld \texttt{value} des erzeugten Objektes speichert.

\begin{slisting}
\Ckw{class} Box(\Cty{int} v) \{\\
\ \ \Cty{int} value = v;\\
\}\\
\\
\Cty{obj} z = Box(x)\\
\Cty{int} i = z.value; \ \ \Ccom{// Auslesen des Feldes} \\
\end{slisting}


Klassen können Variablendeklarationen und Funktionsdefinitionen beinhalten.
Diese werden als \emph{Felddeklarationen} und \emph{Methodendefinitionen} interpretiert.
Der Körper einer Klasse kann auch Anweisungen enthalten.  Solche \emph{Konstruktoranweisungen} werden jedes Mal ausgeführt, wenn ein Objekt der Klasse durch den Konstruktoraufruf erzeugt wird.

Klassen- und Funktionsdefinitionen sind global sichtbar, so daß Klassen und Funktionen sich gegenseitig verwenden können.
Die Sichtbarkeit von Variablen beginnt mit der Zeile nach der Zeile, in der sie definiert sind, und endet, wenn der Block, in dem die Variable deklariert ist, endet.

In AttoL sind Deklarationen syntaktisch Anweisungen.  Es sit also
\emph{syntaktisch} möglich, Deklarationen zu verschachteln; die
semantische Prüfung des Frontends weist allerdings Programme mit
`inneren Klassen' oder `inneren Funktionen' ab.

\subsection{Anweisungen}
AttoL hat die folgenden Anweisungen:
\begin{itemize}
\item Blöcke mit weiteren Anweisungen, umchlossen von geschweiften Klammern (\texttt{\{ \ldots \}})
\item Deklarationen (Abschnitt~\ref{sec:decl})
\item Zuweisungen, wie \texttt{x := 17;}.  Zuweisungen ändern den Inhalt von Variablen,
  Array-Elementen (Abschnitt~\ref{sec:arrays}), oder Feldern.
\item Leere Anweisungen (Alleinstehendes Semikolon)
\item Bedingte Anweisungen mit oder ohne \Ckw{else}-Klausel:
  \begin{slisting}
    \Ckw{if} (x > 0) print(x); \Ckw{else} print(0 - x);\\
  \end{slisting}
\item Logische Prä-Test-Schleifen:
    \begin{slisting}
      \Ckw{while} (x > 0) \{ print(x); x := x - 1; \}
    \end{slisting}
\item \Ckw{break}, das die innerste Schleife abbricht
\item \Ckw{continue}, das über den Rest des Körpers der innersten Schleife springt
\item \Ckw{return}-Anweisungen innerhalb von Funktionen und Methoden beenden die Ausführung der Funktion oder Methode
  und liefern ggf. einen Rückgabewert an den Aufrufer
\item Beliebige \emph{Ausdrücke} (Abschnitt~\ref{sec:expr}).  Diese Ausdrücke werden ausgewertet;
  ihre Werte werden dann verworfen.  Damit können die Seiteneffekte von Funktionsaufrufen, Methodenaufrufen, und
  Konstruktoraufrufen genutzt werden.
\end{itemize}
Bedingte Ausdrücke und Schleifen hängen von der logischen Bedeutung der `Wahrheit' ab, die wir im Abschnitt~\ref{sec:truth}
erklären.

\subsection{Eingebaute Funktionen}
AttoL definiert drei eingebaute Funktionen:
\begin{itemize}
\item \texttt{print(\Cty{obj} x)}, das das Objekt \texttt{x} ausdruckt.
\item \texttt{assert(\Cty{int} x)}, das sicherstellt, daß \texttt{x} logisch `wahr' ist (Section~\ref{sec:truth})  und sonst das Programm abbricht.
\item \texttt{magic(\Cty{obj} x)}, das keine definierte Semantik hat.  Diese Operation erlaubt es uns, mit dem Laufzeitsystem zu experimentieren.
\end{itemize}

Weiterhin definiert AttoL die eingebaute Methode \texttt{size()} für Strings und Arrays.  Für Strings
bestimmt die Methode, wieviele Zeichen in der Zeichenkette sind, und für Arrays bestimmt sie, wieviel Elemente in dem Array gehalten werden.

\subsection{Ausdrücke}\label{sec:expr}
Es gibt verschiedene gültige Ausdrücke:
\begin{itemize}
\item Ganzzahl-Literale  (\textit{int}) wie \texttt{17} oder \texttt{-42}
\item Zeichenketten-Literale in Anführungszeichen (\textit{string}), wie z.B.\@ \texttt{\"{}Hallo\"}
\item Bezeichner (\textit{id}) wie z.B. der Name \texttt{print}
\item Das `kein-Objekt'-Literal \Ckw{NULL}.
\item Ausdrücke, die in Klammern gepackt sind: \texttt{( \ldots )}
\item Arithmetische Ausdrücke, die sich aus den Operatoren \texttt{+}, \texttt{-}, \texttt{*}, \texttt{/} ergeben.
  Diese sind für Integer-Operanden definiert, mit der üblichen Semantik.
\item Array-Ausdrücke (Section~\ref{sec:arrays}).
\item Logische Ausdrücke (Section~\ref{sec:truth}).
\item Typvergleiche der Form \texttt{expr \Ckw{is} \Cty{ty}}, wobei \Cty{ty} einer der folgenden Typen sein kann: \Ckw{int}, \Ckw{obj},
  einer der eingebauten Typen, or ein benutzerdefinierter Typ.  Diese Vergleiche lieferen ein `wahres' Ergebnis (Abschnitt~\ref{sec:truth}) gdw
  \texttt{expr} zu einem Objekt des angegebenen Typs auswertet.  Wenn \texttt{expr} \Ckw{NULL} ist, ist die Antwort immer \emph{false}.
\item Ein Funktions- oder Konstruktoraufruf, z.B.\@ \texttt{add(2, 3)} oder \texttt{Box(7)}.
\item Feldzugriff, Notation \texttt{expr.field}; z.B.\@ \texttt{box.value}.
\item Methodenaufrufe wie z.B.\@ \texttt{"string".size()}.
\end{itemize}

\subsubsection{Array-Ausdrücke}\label{sec:arrays}
Arrayobjekte können \emph{erzeugt}, \emph{ausgelesen}, und \emph{aktualisiert} werden.
Es gibt zwei syntaktische Formen zur Erzeugung eines Arrays:

\begin{slisting}
\Cty{obj} arr1 = [1, 2, 3];\\
\Cty{obj} arr2 = [/ 100];\\
\Cty{obj} arr3 = [1, 2, 3, / 100];\\
\end{slisting}

\texttt{arr1} wird mit einem ein Array der Größe 3 mit den drei Elementen \texttt{1}, \texttt{2}, \texttt{3} (in dieser Reihenfolge) initialisiert.
\texttt{arr2} wird ein Array der Größe 100 sein, in dem alle Elemente auf \Ckw{NULL} gesetzt sind.
\texttt{arr3} wird ebenfalls ein Array der Größe 100, und alle Elemente \emph{außer den ersten drei Elementen} werden auf \Ckw{NULL} gesetzt.
Die ersten drei Elemente werden mit \texttt{1}, \texttt{2}, \texttt{3} (in dieser Reihenfolge) initialisiert.

Um ein Array \emph{auszulesen} oder zu \emph{aktualisieren}, kann man mit der Index-Notation (eckige Klammern) auf Array-Elemente zugreifen:
\begin{slisting}
arr1[0] := arr1[1];
\end{slisting}

\subsubsection{Boolesche Ausdrücke und Wahrheit}\label{sec:truth}
AttoL hat keinen dedizierten booleschen Typ.  Ein AttoL-Wert ist \emph{wahr} gdw der Wert weder
0 noch \Ckw{NULL} ist, und sonst \emph{nicht wahr}.  Wahrheitswerte werden insbesondere von Vergleichen berechnet:
\begin{itemize}
\item Gleichheitsprüfung, die die \texttt{==} (gleich) and \texttt{!=} (ungleich)-Operatoren verwendet
\item Numerische Vergleiche, die \texttt{<} (kleiner-als), \texttt{<=} (kleiner-gleich),
   \texttt{>} (größer-als) und \texttt{>=} (größer-gleich)-Operatoren auf \Cty{int}-Werten verwenden
\end{itemize}
sowie den unären Präfixoperator \Ckw{not} zur logischen Negation., 

\subsection{Seiteneffekte}
Seiteneffekte in einer Folge von Befehlen werden von vorne nach hinten und von links nach rechts ausgeführt, mit den folgenden Ausnahnmen:
\begin{itemize}
\item Seiteneffekte in Funktionen/Methoden werden ausgelöst, wenn die Funktion/Methode aufgerufen wird, bis zur beendenden \Ckw{return}-Anweisung oder dem Ende der Funktion/Methode.
\item Seiteneffekte in bedingten Anweisungen und Schleifen werden nur bedingt ausgeführt (und im Falle von Schleifen ggf.\@ wiederholt).
\item Seiteneffekte in Klassendefinitionen (außerhalb von Methoden) werden ausgeführt, wenn die Klasse durch einen Konstruktoraufruf instanziiert wird.
\end{itemize}


%% sed 's/\([^,]\)\*/\1\\star/' | awk '{ l = $2; m= $3;  rest=""; start = 4; if (l == "|") {m = "\\VB"; l = ""; start=3} ; for (i = start; i <= NF; i++) { r = $i; if (r != "\\star") { if (i > start) { rest = rest "\\ "}; p = "\\nt"; k = substr(r, 0, 1); r = substr(r, 2); if (k == "\\") {r = "\\" r} else if (k == "%") { p = "\\varepsilon"; } else if (k == ",") { p = "\\vterminal"} else if (k == ":") {p = "\\terminal"} else { p = "\\nt"; r = k r}; if (r == "{" || r == "}") {r = "\\" r}; } else {p = ""} rest = rest " " p "{" r "}"; } ; if (l != "") {l = "\\nta{" l "}"};  printf "%s & %s & %s \\\\\n", l, m, rest}'

%% PROGRAM ::= STMT *
%% TY ::= ,int
%%      | ,obj
%% FORMAL ::= TY :id
%% FORMALS ::= %
%%           | FORMAL
%%           | FORMALSL
%% FORMALSL ::= FORMAL
%%            | FORMAL ,, FORMALSL
%% ACTUALS ::= %
%%           | EXPR
%%           | ACTUALSL
%% ACTUALSL ::= EXPR
%%            | EXPR ,, ACTUALSL
%% BLOCK ::= ,{ STMT * ,}
%% STMT ::= BLOCK
%%        | TY :id ,;
%%        | TY :id = EXPR ,;
%%        | EXPR ,;
%%        | EXPR ,:= EXPR ,;
%%        | ,;
%%        | ,if EXPR STMT ,;
%%        | ,if EXPR STMT ,else STMT
%%        | ,while ,( EXPR ,) STMT
%%        | ,break ,;
%%        | ,continue ,;
%%        | ,return ,;
%%        | ,return EXPR ,;
%%        | TY :id ,( FORMALSL ,) BLOCK
%%        | ,class :id ,( FORMALSL ,) BLOCK
%% EXPR ::= ,not EXPR
%%        | EXPR BINOP EXPR
%%        | EXPR ,is ,id
%%        | EXPR ,is TY
%%        | EXPR ,. EXPR
%%        | EXPR ,( ACTUALSL ,)
%%        | EXPR ,[ EXPR ,]
%%        | ,[ ACTUALSL ,]
%%        | ,[ ACTUALSL / EXPR ,]
%%        | :int
%%        | :string
%%        | ,NULL
%%        | :real
%%        | :id
%%        | ,( EXPR ,)
%% BINOP ::= ,==
%%         | ,!=
%%         | ,<
%%         | ,>
%%         | ,<=
%%         | ,>=
%%         | ,+
%%         | ,-
%%         | ,*
%%         | ,/

\begin{figure}
\[
\begin{array}{lcl}
  \nta{PROGRAM} & ::= &  \nt{STMT} {\star} \\
\nta{TY} & ::= &  \vterminal{int} \\
 & \VB &  \vterminal{obj} \\
\nta{FORMAL} & ::= &  \nt{TY}\  \terminal{id} \\
\nta{FORMALS} & ::= &  \varepsilon{} \\
 & \VB &  \nt{FORMAL} \\
 & \VB &  \nt{FORMALSL} \\
\nta{FORMALSL} & ::= &  \nt{FORMAL} \\
 & \VB &  \nt{FORMAL}\  \vterminal{,}\  \nt{FORMALSL} \\
\nta{ACTUALS} & ::= &  \varepsilon{} \\
 & \VB &  \nt{EXPR} \\
 & \VB &  \nt{ACTUALSL} \\
\nta{ACTUALSL} & ::= &  \nt{EXPR} \\
 & \VB &  \nt{EXPR}\  \vterminal{,}\  \nt{ACTUALSL} \\
\nta{BLOCK} & ::= &  \vterminal{\{}\  \nt{STMT} {\star}\  \vterminal{\}} \\
\nta{STMT} & ::= &  \nt{BLOCK} \\
 & \VB &  \nt{TY}\  \terminal{id}\  \vterminal{;} \\
 & \VB &  \nt{TY}\  \terminal{id}\  \nt{=}\  \nt{EXPR}\  \vterminal{;} \\
 & \VB &  \nt{EXPR}\  \vterminal{;} \\
 & \VB &  \nt{EXPR}\  \vterminal{:=}\  \nt{EXPR}\  \vterminal{;} \\
 & \VB &  \vterminal{;} \\
 & \VB &  \vterminal{if}\  \nt{EXPR}\  \nt{STMT}\  \vterminal{;} \\
 & \VB &  \vterminal{if}\  \nt{EXPR}\  \nt{STMT}\  \vterminal{else}\  \nt{STMT} \\
 & \VB &  \vterminal{while}\  \vterminal{(}\  \nt{EXPR}\  \vterminal{)}\  \nt{STMT} \\
 & \VB &  \vterminal{break}\  \vterminal{;} \\
 & \VB &  \vterminal{continue}\  \vterminal{;} \\
 & \VB &  \vterminal{return}\  \vterminal{;} \\
 & \VB &  \vterminal{return}\  \nt{EXPR}\  \vterminal{;} \\
 & \VB &  \nt{TY}\  \terminal{id}\  \vterminal{(}\  \nt{FORMALSL}\  \vterminal{)}\  \nt{BLOCK} \\
 & \VB &  \vterminal{class}\  \terminal{id}\  \vterminal{(}\  \nt{FORMALSL}\  \vterminal{)}\  \nt{BLOCK} \\
\nta{EXPR} & ::= &  \vterminal{not}\  \nt{EXPR} \\
 & \VB &  \nt{EXPR}\  \nt{BINOP}\  \nt{EXPR} \\
 & \VB &  \nt{EXPR}\  \vterminal{is}\  \vterminal{id} \\
 & \VB &  \nt{EXPR}\  \vterminal{is}\  \nt{TY} \\
 & \VB &  \nt{EXPR}\  \vterminal{.}\  \nt{EXPR} \\
 & \VB &  \nt{EXPR}\  \vterminal{(}\  \nt{ACTUALSL}\  \vterminal{)} \\
 & \VB &  \nt{EXPR}\  \vterminal{[}\  \nt{EXPR}\  \vterminal{]} \\
 & \VB &  \vterminal{[}\  \nt{ACTUALSL}\  \vterminal{]} \\
 & \VB &  \vterminal{[}\  \nt{ACTUALSL}\  \nt{/}\  \nt{EXPR}\  \vterminal{]} \\
 & \VB &  \terminal{int} \\
 & \VB &  \terminal{string} \\
 & \VB &  \vterminal{NULL} \\
 & \VB &  \terminal{id} \\
 & \VB &  \vterminal{(}\  \nt{EXPR}\  \vterminal{)} \\
\nta{BINOP} & ::= &  \vterminal{==} 
  \VB   \vterminal{!=} 
  \VB   \vterminal{<} 
  \VB   \vterminal{>} 
  \VB   \vterminal{<=} 
  \VB   \vterminal{>=} 
  \VB   \vterminal{+} 
  \VB   \vterminal{-} 
  \VB   \vterminal{*} 
  \VB   \vterminal{/} \\
\end{array}
\]
\caption{EBNF-Syntax für AttoL.  $\nt{A}\star$ steht hier für null oder mehr Wiederholungen von $\nt{A}$. }
\label{fig:syntax}
\end{figure}

\section{AttoVM: AttoL-Übersetzer und Laufzeitumgebung}

\subsection{Module}\label{a:modules}

AttoVM verwendet mehrere Module, von denen einige von Python-Skripten erzeugt werden.  Die meisten Module
verwenden jeweils eine gleichnahmige Header-Datei (z.B. \texttt{ast.h} für \texttt{ast.c}).

\subsubsection{Frontend}

Mehrere Module im Frontend werden von dem Skript \texttt{mk-parser.py} erzeugt; diese sind mit einem  Stern (\textbf{*}) markiert.

\begin{tabular}{p{4cm}p{12cm}}
\texttt{lexer.l}\textbf{*} & Der Lexer wird vom Programm \texttt{flex} in ein C-Programm umgesetzt.  Er erkennt die Tokens/Lexeme von
Eingabeprogrammen und meldet diese an den Parser. \\
\texttt{parser.c}\textbf{*} & Der Parser liest die Lexeme des Lexers und versucht, diese in den abstrakten Syntaxbaum abzubilden. \\
\texttt{ast.c} & Der abstrakte Syntaxbaum repräsentiert das Programm nach dem Parsen und während der Programmanalysen.  Dieses Modul stellt Operationen zum Erzeugen
und Deallozieren von Baumknoten zur Verfügung.
\\
\texttt{builtins.c} & Dieses Modul definiert die fest eingebauten Typen und Operationen. \\
\texttt{class.c} & Dieses Modul definiert das Konzept einer `Klasse' in AttoL.  Auch eingepackte \texttt{int}-Werte haben eine zugeordnete Klasse. \\
\texttt{symbol-table.c} & Die Symboltabelle bildet Bezeichnernummern auf Symboltabelleneinträge ab.  Symboltabelleneinträge werden durch
das gesamte System hindurch verwendet; sie geben jedem eingebauten oder benutzerdefinierten Bezeichner wichtige Kontextinformationen (z.B. Typinformationen, Parameter für Funktionsdefinitionen, Art des Symbols usw.). \\
\texttt{unparser.c}\textbf{*} & Der \emph{Unparser} enthält Code, der einen abstrakten Syntaxbaum ausdruckt.\\
\end{tabular}

%% Several of the frontend modules are generated by the script \texttt{mk-parser.py}; these are marked by an asterisk (\textbf{*}) below.

%% \begin{tabular}{p{4cm}p{12cm}}
%% \texttt{lexer.l}\textbf{*} & The lexer is translated into a C program by another program, \texttt{flex}.
%% This code detects tokens and lexemes and reports them to the parser. \\
%% \texttt{parser.c}\textbf{*} & The parser identifies tokens and lexemes and attempts to construct an
%% abstract syntax tree from them. \\
%% \texttt{ast.c} & The abstract syntax tree (AST) represents the program after parsing and during program analysis.
%% This module also provides operations for creating and deallocating AST nodes.\\
%% \texttt{builtins.c} & This module contains built-in types and operations. \\
%% \texttt{class.c} & This module defines the notion of an AttoL `class'.
%% Boxed \texttt{int} values have a separate class defined therein. \\
%% \texttt{symbol-table.c} & The symbol table maps identifiers (internally stored as numbers) to \emph{symbol table entries}.
%% Symbol table entries store the bindings for each identifier, and are used throughout the system.  The
%% symbol table stores human-readable names, types, storage locations, function arities and other pertinent information.\\
%% \texttt{unparser.c}\textbf{*} & The \emph{unparser} contains code for printing out the abstract syntax tree.\\
%% \end{tabular}

\subsubsection{Analysen}

Die Analysen sind in \texttt{analyses.h} zusammengefaßt.  Die Analysen werden vom Frontend und teilweise vom Middle-End verwendet.
Sie nutzen die Symboltabelle intensiv.

\begin{tabular}{p{4cm}p{12cm}}
\texttt{name-analysis.c} & Die Namensanalyse identifiziert Namen im Eingabeprogramm und sorgt dafür, daß zusammengehörende Namen den gleichen Eintrag in der Symboltabelle haben.\\
\texttt{type-analysis.c} & Die Typanalyse stellt sicher, daß das Programm korrekt getypt ist.  Sie führt notwendige Typkonversionen (Einpacken/Auspacken) ein und
strukturiert Teile des abstrakten Syntaxbaumes, die sich auf Typen beziehen, um. \\
\texttt{control-flow-graph.c} & Kontrollfluß-Analyse.  Diese Analyse untersucht die Programmstruktur und übersetzt sie in einen Graphen. \\
\texttt{data-flow.c} & Allgemeine Datenfluß-Analyse.  Diese Analyse nimmt den Kontrollflußgraphen und analysiert das Programm, um verschiedene Optimierungen oder Fehlerprüfungen durchzuführen. \\
\texttt{data-flow-definite-assignments.c} & `Definite-Assignments'-Analyse.  Prüft, daß niemand eine Variable liest, die noch nicht beschrieben wurde.\\
\texttt{data-flow-$\cdots$.c} & Middle-End: Weitere Programmanalysen, die zur Optimierung genutzt werden. \\
\texttt{symint.c} & Middle-End: Zahlenrepräsentierung, die zur Optimierung genutzt wird. \\
\end{tabular}
%% \texttt{analyses.h} captures all current program analyses.
%% They are part of the frontend and make intensive use of the symbol table.

%% \begin{tabular}{p{4cm}p{12cm}}
%% \texttt{name-analysis.c} & Name analysis identifies names in the input program and resolves names according to the scoping rules.  If name analysis succeeds, all names share appropriate symbol table entries.\\
%% \texttt{type-analysis.c} & Type analysis ensures that the program is correctly typed.  It adds implicit coercions (ad-hoc polymorphism) and
%% restructures some parts of the AST that relate to types. \\
%% \texttt{storage-allocation.c} & Storage allocation assigns storage (global, stack, or heap) to variables. \\
%% \end{tabular}

\subsection{Backend}

Das Assembler-Modul (mit einem  Stern (\textbf{*}) markiert) wird von einem Skript \texttt{mk-codegen.py} erzeugt.

\begin{tabular}{p{4cm}p{12cm}}
\texttt{assembler-buffer.c} & Assembler buffers are mechanisms for writing out assembly code.  They further define the concept of a jump label (\Cty{label\_t}), which can be resolved at a later time.
This module cooperates with \texttt{assembler.c} to define a disassembler. \\
\texttt{assembler.c}\textbf{*} & This module contains all supported assembly instructions. \\
\texttt{address-store.c} & The address store is used to support disassembly.  It is a hash table that records known function names and mentions them during disassembly, to increase readability. \\
\texttt{registers.c} & The register module defines the available registers.  \\
\texttt{baseline-backend.c} & The actual compiler, which takes the AST and emits assembly instructions to an assembly buffer. \\
\texttt{backend-test.c} & A test suite for the compiler. \\
\end{tabular}

\subsection{Laufzeitsystem}
\begin{tabular}{p{4cm}p{12cm}}
\texttt{runtime.c} & Ein generisches Laufzeit-Unterstützungsmodul. \\
\texttt{object.c} & Objektallozierung. \\
\texttt{dynamic-compiler.c} & Der dynamische Übersetzungs- und Optimierungscode \\
\texttt{stackmap.c} & Stapelkarten, für die automatische Speicherverwaltung. \\
\texttt{debugger.c} & Ein interaktiver Assembler-Debugger, der mit dem Parameter \texttt{-f debug-asm} aktiviert werden kann (nur auf Linux) \\
\texttt{heap.c} & Ablagespeicher und automatische Speicherverwaltung \\
\end{tabular}

\subsubsection{Unterstützung}

\begin{tabular}{p{4cm}p{12cm}}
  \texttt{atl.c} & Hauptprogramm. \\
\texttt{chash.c} & Eine generische Hashtabellenimplementierung. \\
\texttt{bitvector.c} & Eine effiziente Implementierung von Bit-Vektoren mit beliebiger Länge. \\
\texttt{cstack.c} & Eine generische Stapelimplementierung. \\
\texttt{timer.c} & Eine Bibliothek zur Zeitmessung. \\
\end{tabular}


\section{Abstrakter Syntaxbaum von AttoVM}\label{a:ast}

Der abstrakte Syntaxbaum von AttoVM besteht aus verschiedenen Knotentypen.  In dieser Übung verwenden wir die wichtigsten davon.
Der Baum hat zwei Arten von Knoten: \emph{Wertknoten} und \emph{Zweigknoten}.  Erstere beinhalten Werte, letztere können
Kindknoten beinhalten.

Tabelle~\ref{fig:valnodes} beschreibt die Wertknoten, und Tabelle~\ref{fig:rnodes} die Zweigknoten.

Die komplette Liste ist in \texttt{ast.h} definiert.

Wenn Sie den abstrakten Syntaxbaum ausgeben, werden u.U. noch
zusätzliche Attribute der Baumknoten ausgegeben (mit \texttt{\#}
markiert).  Die für uns relevanten Attribute sind folgende:

\begin{itemize}
\item   \texttt{\#INT}: Integer-Zahl
\item   \texttt{\#OBJ}: Objekt
\item   \texttt{\#LVALUE}: LValue (zuweisbarer Wert
\item   \texttt{\#DECL}: Bezeichner wird an dieser Stelle deklariert
\item   \texttt{\#CONST}: Bezeichner wurde als `konstant' (\texttt{const}) deklariert
\end{itemize}

Alle wesentlichen Berechnungen werden durch \textsf{FUNAPP}-Knoten dargestellt.  Unten sehen Sie z.B.
den abstrakten Syntaxbaum des Programmes \texttt{print(1)} nach der Namensanalyse:

\begin{verbatim}
  (FUNAPP#OBJ
    (SYM[-12] print):#OBJ
    (ACTUALS
      (FUNAPP#OBJ
        (SYM[-3] *convert):#OBJ
        (ACTUALS
          1:INT#INT
        )
      )
    )
  )
\end{verbatim}

Dies zeigt uns, daß auf die Zahl $1$ (Typ-Markierung \texttt{\#INT}) zunächst die Funktion
\texttt{*convert} mit Symboltabelleneintrag $-3$ angewendet wird.  \texttt{*convert} ist eine wichtige Funktion, die Typkonversionen
(insbesondere Einpacken/Auspacken) durchführt.  Der Ergebnistyp ist in dem Funktionsaufruf mit angegeben:
\begin{verbatim}
        (SYM[-3] *convert):#OBJ
\end{verbatim}
Es wird also ein \texttt{\#OBJ} zurückgeliefert.  Das Ergebnis dieses Funktionsaufrufes ist nun wiederum ein Parameter
für einen Funktionsaufruf der eingebauten Funktion \texttt{print}:
\begin{verbatim}
    (SYM[-12] print):#OBJ
\end{verbatim}
Diese druckt den Parameter aus (und liefert \Ckw{NULL} zurück).

\begin{figure}[h]
\begin{tabular}{|p{3cm}|p{14cm}|}
\hline
\textbf{Knoten} & \textbf{Inhalt} \\
\hline
\hline
\textsf{INT} &  64-Bit-Integer-Zahl\\
\hline
\textsf{STRING} & Zeichenkette \\
\hline
\textsf{NAME  } & Name (wird bei Namensanalyse in \textsf{ID} umgewandelt) \\
\hline
\textsf{ID    } & Bezeichner: eine Zahl, die in die Symboltabelle deutet.  Negative Zahlen sind eingebaute Bezeichner.  \\
\hline
\end{tabular}
\caption{Wertknoten mit ihren Wert-Inhalten}\label{fig:valnodes}
\end{figure}

\begin{figure}[h]
\begin{tabular}{|p{2.5cm}|p{2cm}|p{12cm}|}
\hline
\textbf{Knoten} & \textbf{Kinder} & \textbf{Bedeutung} \\
\hline
\hline
\textsf{NULL}		& ---		& Der konstante Wert \Ckw{NULL}.\\
\hline
\textsf{ISPRIMTY}	& $(e)$		& Primitiver Typtest; wird von der Typanalyse in \textsf{ISINSTANCE} oder \textsf{INT}-Konstante umgeschrieben. \\
\hline
$\textsf{ISINSTANCE}^{\dagger}$	& $(e, \textit{id})$ & Typüberprüfung: hat der Ausdruck $e$ zur Laufzeit den Typ, der vom Bezeichner \textit{id} angegeben wird? \\
\hline
\textsf{ARRAYLIST}	& $(e_0, \ldots, e_n)$ & Initialisierungsliste für Array. \\
\hline
\textsf{ARRAYVAL}	& $(\textit{al}, e)$ & Erzeuge ein Array mit initialen Werten \textit{al} (immer \textsf{ARRAYLIST}) der Größe $e$ (optional; wenn \Cpp{NULL}, wird Länge von \textit{al} genommen). \\
\hline
\textsf{ARRAYSUB}	& $(e_0, e_1)$	& Zugriff auf Eintrag $e_1$ im Array $e_0$ ($\texttt{e}_0$[$\texttt{e}_1$]). \\
\hline
\textsf{ACTUALS}	& $(e_0, \ldots, e_n)$ & Parameterliste für Funktionsaufruf. \\
\hline
\textsf{FUNAPP}		& $(\textit{id}, \textit{a})$ & Führt die Funktion \textit{id} mit den Parametern der Parameterliste \textit{a} (immer \textsf{ACTUALS}) aus. \\
\hline
$\textsf{NEWINSTANCE}^\dagger$	&$(\textit{id}, \textit{a})$ & Konstruktoraufruf, wobei \textit{id} der Klassenname ist und \textit{a} die Parameterliste.\\
\hline
$\textsf{METHODAPP  }^\dagger$	&$(e, \textit{id}, \textit{a})$ & Methodenaufruf der Methode \textit{id} auf dem von $e$ berechneten Objekt, mit Parametern (immer \textsf{ACTUALS}) in \textit{a} \\
\hline
\textsf{BLOCK}		& $(s_0, \ldots, s_n)$ & Führe alle $s_i$ hintereinander aus. \\
\hline
\textsf{SKIP} 		& ---		& Keine Aktion. \\
\hline
\textsf{IF}		& $(e, s_t, s_f)$ & Wertet Ausdruck $e$ auf Integer-Zahl aus.  Wenn nicht-Null, wird $s_t$ ausgeführt, sonst $s_f$.  $s_f$ kann \Cpp{NULL} sein (in diesem Fall wird nichts ausgeführt). \\
\hline
\textsf{WHILE}		& $(e, s)$	& Werte $e$ auf Integer-Zahl aus.  Wenn nicht $0$, führe $s$ aus.  Wiederhole, bis $e$ auf $0$ auswertet. \\
\hline
\textsf{BREAK}		& ---		& \texttt{break} aus einer Schleife heraus. \\
\hline
\textsf{CONTINUE} 	& ---		& \texttt{continue}: Rest des Schleifenkörpers überspringen, weiter in der Schleife. \\
\hline
\textsf{VARDECL}	& $(\textit{id}, e)$ & Deklaration einer Variablen \textit{id}.  Kindknoten $e$ ist die optionale Initialisierung der Variablen, sonst \Cpp{NULL}. \\
\hline
\textsf{ASSIGN}		& $(e_0, e_1)$  & Weist Variable oder Arrayeintrag  $e_0$ den Wert $e_1$ zu. $e_0$ muß ein LValue sein (s.u.). \\
\hline
\textsf{MEMBER}		& $(e, \textit{id}) $ & Feldzugriff auf Feld \textit{id} des von $e$ berechneten Objektes. \\ 
\hline
\textsf{FORMALS    }	&$(\textit{id}_0, \ldots, \textit{id}_n)$ & Eine Folge von Bezeichnern mit Typattributen, die als Liste formaler Parameter dient.\\
\hline
\textsf{FUNDEF     }	&$(\textit{id}, \textit{f}, s)$ & Funktionsdefinition für Funktion mit Namen \textit{id} mit Parametern \textit{f} (immer \textsf{FORMALS}) und Funktionskörper $s$ (immer ein \textsf{BLOCK})\\
\hline
\textsf{CLASSDEF   }	&$(\textit{id}, \textit{f}, s, c)$ & Klassendefinition für Klasse mit Namen \textit{id} und Konstruktorparametern \textit {f} (\textsf{FORMALS}).  Der Klassenkörper $s$ ist ein \textsf{BLOCK}.  $c$ ist nach dem Parsen zunächst \Cpp{NULL}, wird während der Typanalyse aber mit einem Konstruktor (\textsf{FUNDEF}) gefüllt $c$.\\
\hline
\end{tabular}
\caption{Wertknoten mit ihren Wert-Inhalten.  Ausdrücke (Berechnungen etc.) werden im Wesentlichen durch \textsf{FUNAPP} repräsentiert.  LValues sind
  entweder ein \textsf{ID} aus Abbildung~{fig:valnodes} oder ein \textsf{ARRAYSUB}.
  AST-Knoten, die mit ($\dagger$) markiert sind, werden nicht vom Parser, sondern von der Typanalyse erzeugt.
}\label{fig:rnodes}

\end{figure}

\subsection{Erzeugen von 2OMP-Code mit \texttt{assembler.c}}

Zur Code-Erzeugung im Compiler-Backend stehen Ihnen Funktionen zur Verfügung, die Maschinenspracheinstruktionen an einen
\emph{Maschinencode-Puffer} (per Konvention \texttt{buf} genannt) anhängen.  Um z.B.\@ das aktuelle \texttt{\$v0}-Register um $3$ zu erhöhen, würden
Sie schreiben:

\texttt{emit\_addiu(buf, REGISTER\_T0, 3);}

Um einen Ihnen bereits bekannten Zeiger \texttt{addr} in Register \texttt{\$t1} zu laden, könnten Sie wiederum schreiben:

\texttt{emit\_la(buf, REGISTER\_T1, addr);}

Die vollständige Befehlsliste finden Sie in \texttt{assembler.h}, und eine Beschreibung der (MIPS-artigen) Befehle in der 2OPM-Doukmentation.

\subsubsection{Sprungmarken}

Einige der 2OPM-Operationen benötigen Sprungmarken.  Falls Sie einen solchen Befehl verwenden wollen,
stehen Sie vor der Herausforderung, daß Sie beim Schreiben des Befehls oft noch nicht die genaue Adresse kennen, an die Sie springen wollen.
Um dieses Problem zu lösen, verwendet AttoVM Sprungmarkenvariablen (vom Typ \Cty{label\_t}).


Wenn Sie den entsprechenden \texttt{emit\_}-Befehl ausführen, übergeben Sie einen Zeiger auf eine von Ihnen angelegte lokale
Variable vom Typ \Cty{label\_t}:

%% To utilise such a label, first invoke the branch function you want to use, passing a pointer to
%% storage that you have allocated for the label (typically a stack-dynamic variable).  For example:

\vspace{0.5cm}
\begin{tabular}{ll}
&\texttt{\Cty{label\_t} label;}\\
$A$&\texttt{emit\_beqz(buf, REGISTER\_T0, \&label);}\\
\end{tabular}
\vspace{0.5cm}

Sie können (und müssen!) dann zu einem beliebigen späteren Zeitpunkt
über diese Variable die korrekte Sprungzieladresse über die Funktion \texttt{buffer\_setlabel2(\&label, buf)} einsetzen; als
Fortsetzung zum obigen Beispiel:

\vspace{0.5cm}

\begin{tabular}{ll}
&\texttt{emit\_addi(buf, REGISTER\_T0, 3);} \\
$B$&\texttt{buffer\_setlabel2(\&label, buf);} \\
&\texttt{emit\_la(buf, REGISTER\_T1, addr);} \\
\end{tabular}

\vspace{0.5cm}

In diesem Beispiel teilen Sie dem Backend mit, daß Sie (falls die gegebene Bedingung erfüllt ist) von $A$ nach $B$ springen wollen.

Um rückwärts zu springen, können Sie \texttt{buffer\_target()} und \texttt{buffer\_setlabel()} verwenden:

\vspace{0.5cm}

\begin{tabular}{ll}
  &\texttt{emit\_addi(buf, REGISTER\_T0, 3);} \\
  $B$&\texttt{\Cty{void *}addr = buffer\_target(buf);} \\
  $C$ &\texttt{emit\_li(buf, REGISTER\_T1, 23);} \\
  & \ldots \\
  &\texttt{\Cty{label\_t} label;}\\
  $A$&\texttt{emit\_beqz(buf, REGISTER\_T0, \&label);}\\
  & \texttt{buffer\_setlabel(\&label, addr);}\\
\end{tabular}

\vspace{0.5cm}

In diesem Beispiel springen wir von  $A$ nach $C$, wenn die Bedingung wahr ist.
$B$ extrahiert hier die Speicheradresse des Assemblerbefehls, zu der wir springen wollen (also die in $C$ geschriebene Instruktion).
Wir verwenden diese Adresse dann beim Aufruf von \texttt{buffer\_setlabel()}.




\vspace{0.5cm}

\begin{tabular}{ll}
&\texttt{\Cty{label\_t} label;}\\
$A$&\texttt{emit\_beqz(buf, REGISTER\_T0, \&label);}\\
&\ldots \\
&\texttt{emit\_addiu(buf, REGISTER\_T0, 3);} \\
$B$&\texttt{buffer\_setlabel2(\&label, buf);} \\
&\texttt{emit\_la(buf, REGISTER\_T1, addr);} \\
\end{tabular}

\vspace{0.5cm}



Sie können in Ihrem Code die folgenden Register verwenden:

\vspace{0.5cm}

\begin{tabular}{|l|l|}
\hline
\textbf{Register} & \textbf{Präprozessormakro mit Registernummer} \\
\hline
  \texttt{\$t0} & \texttt{REGISTER\_T0} \\
\hline
  \texttt{\$t1} & \texttt{REGISTER\_T1} \\
\hline
  \texttt{\$v0} & \texttt{REGISTER\_V0} \\
\hline
  \texttt{\$gp} & \texttt{REGISTER\_GP} (Statische Variablen) \\
\hline
  \texttt{\$sp} & \texttt{REGISTER\_SP} (Stapelzeiger) \\
\hline
\end{tabular}

\end{document}

\subsection{Emitting 2OPM Code}\label{a:2omp}

\subsubsection{Generating 2OPM assembly with \texttt{assembler.c}}

The \texttt{assembler.h}/\texttt{assembler.c} library provides
facilities for emitting assembly code.  This assembly code must be
written to a \emph{machine code buffer}, type \Cty{buffer\_t}, from
\texttt{assembler-buffer.h}.  Such a buffer is allocated in a memory region specifically
allocated to allow reading, writing, and execution.

Once you have access to such a buffer, you can e.g. write the following to increment
 register  \texttt{\$v0} by $3$:

\texttt{emit\_addi(buf, REGISTER\_V0, 3);}

To load a known pointer \texttt{addr} into  register \texttt{\$t1}, you can write:

\texttt{emit\_la(buf, REGISTER\_T1, addr);}

The exact list is defined in \texttt{assembler.h}, with the prototypes
matching the 2OPM manual.

\subsubsection{Jump Labels}




\end{document}


