\documentclass[11pt,a4paper]{article} 
\usepackage[margin=2cm]{geometry}
%\usepackage{babel}
\usepackage[utf8]{inputenc}
\usepackage{hyperref}
\usepackage{tikz}
\usepackage{pgf}
\usepackage{color}
\usepackage{moresize}
\usepackage{xcolor}
\usepackage{listings}
\usepackage{wasysym}
%\usepackage{mips}
\usepackage{pgfpages}
\renewcommand{\theenumi}{\alph{enumi}}
\renewcommand{\theenumii}{\roman{enumii}}
\usetikzlibrary{decorations.pathreplacing, backgrounds, trees, patterns, arrows, shapes, decorations.markings, calc, positioning, fit, matrix, decorations.pathmorphing}
\renewcommand*\ttdefault{txtt}


\newcommand{\mybackslash}{\backslash{}{}{}{}{}{}{}}
\newenvironment{slisting}{%
        \begin{tt}%
        \begin{tabular}{l}%
        }
        {%
        \end{tabular}%
        \end{tt}%
        }

\definecolor{dcyan}{rgb}{0,0.4,0.4}

\newcommand{\Cinclude}[1]{\textcolor{dgreen}{\#include $<$#1$>$}}
\newcommand{\Cty}[1]{\textcolor{dblue}{\texttt{#1}}}
\newcommand{\Ccom}[1]{\textcolor{dgreen}{\texttt{#1}}}
\newcommand{\Cpp}[1]{\textcolor{dcyan}{\texttt{#1}}}
\newcommand{\Ckw}[1]{\textbf{\texttt{#1}}}


\newcommand{\coloneq}{\mathrel{\mathop:}=}
\newcommand{\bincode}[1]{\texttt{\textcolor{blue}{#1}}}
\newcommand{\black}[1]{{\color{black}#1}}
\newcommand{\bincodeB}[1]{\texttt{\textbf{\textcolor{black}{#1}}}}

\newcommand{\ONE}{\texttt{1}}
\newcommand{\ZERO}{\texttt{0}}
\newcommand{\NO}{\node {\ONE};}
\newcommand{\NZ}{\node {\ZERO};}

%% \setbeameroption{show notes}
%% \setbeameroption{show notes on second screen=right}

%\newcommand{\fquote}[1]{\frqq #1\flqq}
\newcommand{\fquote}[1]{\glqq #1\grqq}

\tikzstyle{dflowarrow} = [very thick, rounded corners, double distance=1pt, > = triangle 45 new, arrow head=3mm]
\tikzstyle{choicearrow} = [thick, arrow head=3mm]
\tikzstyle{editarrow} = [thick, decorate, decoration=snake, > = triangle 45 new, arrow head=3mm]

\definecolor{dkgreen}{rgb}{0,0.6,0}
\definecolor{dgreen}{rgb}{0,0.4,0}
\definecolor{dblue}{rgb}{0,0,0.5}
\definecolor{grey}{rgb}{0.5,0.5,0.5}
\definecolor{mauve}{rgb}{0.58,0,0.82}

\newcommand{\Prod}{::=}
\newcommand{\VB}{\ |\ }
\newcommand{\tuple}[1]{\ensuremath{\langle #1 \rangle}}
\newcommand{\nt}[1]{\ensuremath{\tuple{\hspace{-0.02cm}\nta{#1}\hspace{0.02cm}}}}
\newcommand{\sem}[1]{\ensuremath{\llbracket #1 \rrbracket}}
\newcommand{\nta}[1]{\ensuremath{\textit{#1}}}
\newcommand{\terminal}[1]{\textit{#1}}
\newcommand{\vterminal}[1]{\textsf{`}\texttt{#1}\textsf{'}}

%% \lstset{ %
%%   language=C       % the language of the code
%%   }
 
\lstset{ %
  language=C,       % the language of the code
  basicstyle=\small\tt,       % the size of the fonts that are used for the code
%  numbers=left,                   % where to put the line-numbers
  numberstyle=\small\color{grey},  % the style that is used for the line-numbers
  stepnumber=1,                   % the step between two line-numbers. If it's 1, each line 
                                  % will be numbered
  numbersep=5pt,                  % how far the line-numbers are from the code
  backgroundcolor=\color{white},  % choose the background color. You must add \usepackage{color}
  showspaces=false,               % show spaces adding particular underscores
  showstringspaces=false,         % underline spaces within strings
  showtabs=false,                 % show tabs within strings adding particular underscores
%  frame=single,                   % adds a frame around the code
  rulecolor=\color{black},        % if not set, the frame-color may be changed on line-breaks within not-black text (e.g. commens (green here))
  tabsize=4,                      % sets default tabsize to 2 spaces
  captionpos=b,                   % sets the caption-position to bottom
  breaklines=true,                % sets automatic line breaking
  breakatwhitespace=false,        % sets if automatic breaks should only happen at whitespace
  title=\lstname,                 % show the filename of files included with \lstinputlisting;
                                  % also try caption instead of title
  keywordstyle=\color{blue},          % keyword style
  commentstyle=\color{dkgreen},       % comment style
  stringstyle=\color{mauve},         % string literal style
  escapeinside={\%*}{*)},            % if you want to add a comment within your code
  morekeywords={*,eingabe,ausgabe,...},               % if you want to add more keywords to the set
  belowskip=-1em
}

\title{AttoL and AttoVM overview\\ \small{Language revision 0.1}}
\begin{document}
\maketitle

AttoL is a small, statically scoped, dynamically typed imperative
programming language.  AttoL also has limited support for
object-oriented programming, and it is \emph{almost} strongly typed.

\section{Language}

Figure~\ref{fig:syntax} summarises the AttoL language syntax.  Each
AttoL program is a (possibly empty) sequence of statements.  As an
example, consider program~\ref{fig:sieve}, which implements the Sieve
of Eratosthenes in AttoL.

\begin{figure}
\begin{slisting}
\Cty{int} size = 1000;\\
\Cty{int} max = 0;\\
\Cty{obj} sieve = [/size]; \Ccom{// allocate array}\\
\Cty{int} x = 2; \\
\Ckw{while} (x < size) \{ \\
\ \     \Ckw{if} (sieve[x] == \Ckw{NULL}) \{ \Ccom{// empty entry}\\
\ \ \ \        max := x; \\
\ \ \ \	\Cty{int} fill = x + x;\\
\ \ \ \         \Ckw{while} (fill < size) \{\\
\ \ \ \ \ \             sieve[fill] := x;\\
\ \ \ \ \ \             fill := fill + x;\\
\ \ \ \         \}\\
\ \     \}\\
\ \     x := x + 1;\\
\}\\
print(max);
\end{slisting}
\caption{Sieve of Eratosthenes in AttoL}\label{fig:sieve}
\end{figure}

\subsection{Static Types}\label{sec:types}
AttoL has two static types: \Cty{int}, which is the type of signed 64 bit
machine integers, and \Cty{obj}, which is the type of all language objects.
AttoL uses implicit coercions (ad-hoc polymorphism) to convert between
the two types at run-time, so that these types cannot cause static type errors.
For example, the following code declares
two variables and then attempts to update them:

\begin{slisting}
\Cty{obj} s = "string";\\
\Cty{int} i = 7;\\
s := 23;\ \ \  \Ccom{// permitted, successful implicit coercion}\\
i := "foo"; \ \Ccom{// runtime type error}\\
\end{slisting}

\Cty{int} variables are represented directly as integers in memory or
in registers, whereas \Cty{obj} variables are always \emph{boxed}.

The \Cty{obj} type is also \emph{lifted}: a special value \Ckw{NULL}
designates a value that is not an object.

If the initialiser is omitted, the variable is declared but its
contents are undefined.  Accessing the variable may crash the program;
this is the source of AttoL's lack of type safety.

\subsection{Dynamic Types}
AttoL supports several dynamic types:
\begin{itemize}
\item \Cty{int}, which captures 64 bit signed machine integers
\item \Cty{string}, which captures character strings
\item \Cty{Array}, which captures arrays
\item Any user-defined types added through a \Ckw{class} definition
\end{itemize}

\subsection{Declarations and Definitions}\label{sec:decl}
In addition to variable declarations (Section~\ref{sec:types}), AttoL permits two forms of definitions:
\emph{function definitions} and \emph{class definitions}.  

Function definitions follow a C-like syntax:

\begin{slisting}
\Cty{int} add(\Cty{int} a, \Cty{int} b) \{\\
\ \ \Ckw{return} a + b;\\
\}\\
\end{slisting}

This defines a binary function \texttt{add} that adds its two
\Cty{int} parameters and returns the result.
Parameter passing in AttoL is always by value.

Class definitions look similar to function definitions, but start with the keyword \Ckw{class}:

\begin{slisting}
\Ckw{class} Box(\Cty{int} v) \{\\
\ \ \Cty{int} value = v;\\
\}\\
\end{slisting}

The above defines a class \Cty{Box} with the single field \texttt{value}.  You can use the name `\Cty{Box}'
like a function to construct a value of this type.  Parameters after this class name are
the parameters to the constructor.  In this case, we are defining the constructor to take a single
argument \texttt{v}, which the class stores in its field \texttt{value} at class construction time.

Classes may contain variable declarations and function definitions,
which are interpreted as \emph{field} declarations and \emph{method}
definitions, respectively.  Class bodies may contain statements
outside of their class bodies, which are executed once every time an
object of this class is constructed.

Class definitions and function definitions are all global in scope,
permitting classes and functions to reference each other in mutual
recursion.  The scope of variable declarations extends from the
declaration site downwards to the end of the current block.  For class
and function parameters, their scope is the function/class body.

In AttoL, declarations are syntactically statements.  Thus, it is
syntactically possible to nest declarations (e.g., define inner
functions or inner classes), though semantically these constructions
are not supported; the static checker will issue an error message if
it finds nested classes or nested functions.

\subsection{Statements}
Statements may be any of the following:
\begin{itemize}
\item Blocks of further statements, enclosed in curly braces (\texttt{\{ \ldots \}})
\item Declarations (Section~\ref{sec:decl})
\item Assignments, such as \texttt{x := 17;}.  Assignments can update variables,
  array elements (Section~\ref{sec:arrays}), or fields.
\item Empty statements (stand-alone semicolons)
\item Conditional expressions with or without else clause:
  \begin{slisting}
    \Ckw{if} (x > 0) print(x); \Ckw{else} print(0 - x);\\
  \end{slisting}
\item Logical pre-test loops:
    \begin{slisting}
      \Ckw{while} (x > 0) \{ print(x); x := x - 1; \}
    \end{slisting}
\item \Ckw{break}, which terminates the innermost loop
\item \Ckw{continue}, which skips over the rest of the body of the innermost loop
\item \Ckw{return} statements within functions or methods will stop execution of the function or method and optionally return a result
\item Arbitrary expressions, whose results are discarded; this is intended to allow constructor, function, and method invocations.
\end{itemize}
Conditional expressions and loops depend on a logical meaning of `truth', which we discuss in Section~\ref{sec:truth}

\subsection{Built-in Functions}
AttoL defines three built-in functions:
\begin{itemize}
\item \texttt{print(\Cty{obj} x)}, which prints out object \texttt{x}
\item \texttt{assert(\Cty{int} x)}, which asserts that \texttt{x} is logically `true' (Section~\ref{sec:truth}) and otherwise aborts the program
\item \texttt{magic(\Cty{obj} x)}, whose semantics are undefined.  It is reserved for experimenting with the run-time system.
\end{itemize}

It further defines the built-in method \texttt{size()} for strings and arrays, to determine (for strings)
how many characters they contain, and (for arrays) how many elements they hold.

\subsection{Expressions}
Expressions can be literal numbers (\textit{int}), literal strings (\textit{string}), identifiers (\textit{id}),
other expressions wrapped in parentheses, or any of the following composite expressions:
\begin{itemize}
\item The literal `not-an-object' value \Ckw{NULL}.
\item Arithmetic expressions, which are applications of the binary operators \texttt{+}, \texttt{-}, \texttt{*}, \texttt{/};
  these are defined on integer operands to perform the corresponding arithmetic operations.
\item Array expressions (Section~\ref{sec:arrays}).
\item Logical expressions (Section~\ref{sec:truth}).
\item Type comparisons, of the form \texttt{expr \Ckw{is} \Cty{ty}}, where \Cty{ty} is \Ckw{int}, \Ckw{obj},
  any of the built-in types, or any user-defined type.  These yield a `true' result (Section~\ref{sec:truth}) if
  \texttt{expr} evaluates to an object of the specified type.  If \texttt{expr} is \Ckw{NULL}, the answer is \emph{false}.
\item A function invocation or constructor invocation, expressed as e.g. \texttt{add(2, 3)} or \texttt{Box(7)}.
\item Field access, written as \texttt{expr.field}; e.g., \texttt{box.value}.
\item Method invocations, such as \texttt{"string".size()}.
\end{itemize}

\subsubsection{Array Expressions}\label{sec:arrays}
Array objects can be \emph{created}, \emph{subscribed}, and \emph{updated}.  There are two
syntactic forms for creating an array:

\begin{slisting}
\Cty{obj} arr1 = [1, 2, 3];\\
\Cty{obj} arr2 = [/ 100];\\
\Cty{obj} arr3 = [1, 2, 3, / 100];\\
\end{slisting}

Here, \texttt{arr1} will be an array of size 3 with the three elements \texttt{1}, \texttt{2}, \texttt{3} in order.  \texttt{arr2}
will be an array of size 100, with all elements set to \Ckw{NULL}.  \texttt{arr3} will also have size 100,
and all elements are set to \Ckw{NULL} except for the first three, which are set to \texttt{1}, \texttt{2}, and \texttt{3}, respectively.

To \emph{subscribe} or \emph{update} an array, array elements can be picked through bracket notation, as in
\begin{slisting}
arr1[0] := arr1[1];
\end{slisting}

\subsubsection{Boolean Expressions and Truth}\label{sec:truth}
AttoL does not have a dedicated booelan type, and in AttoL, a value is \emph{true} whenever
it is neither 0 nor \Ckw{NULL}.  Truth values can arise specifically from comparisons:
\begin{itemize}
\item Equality comparisons, using the \texttt{==} (is-equal) and \texttt{!=} (is-not-equal) operators
\item Relational comparisons, using the \texttt{<} (less-than), \texttt{<=} (less-than-or-equal),
   \texttt{>} (greater-than) and \texttt{>=} (greater-than-or-equal) operators.
\end{itemize}
as well as from the unary prefix \Ckw{not} operator, which logically negates its argument.

\subsection{Side effects}
Side effects are triggered in the sequence in which statements are specified, top-down, and then left-to-right,
with the following exceptions:
\begin{itemize}
\item Side effects in functions/methods are triggered whenever the function/method is invoked, and cease if the function executes a return statement.
\item Side effects in conditional statements and loops are governed by control flow of these statements and loops.
\item Side effects in the bodies of class definitions (but outside of methods) are invoked whenever the class constructor is invoked.
\end{itemize}


%% sed 's/\([^,]\)\*/\1\\star/' | awk '{ l = $2; m= $3;  rest=""; start = 4; if (l == "|") {m = "\\VB"; l = ""; start=3} ; for (i = start; i <= NF; i++) { r = $i; if (r != "\\star") { if (i > start) { rest = rest "\\ "}; p = "\\nt"; k = substr(r, 0, 1); r = substr(r, 2); if (k == "\\") {r = "\\" r} else if (k == "%") { p = "\\varepsilon"; } else if (k == ",") { p = "\\vterminal"} else if (k == ":") {p = "\\terminal"} else { p = "\\nt"; r = k r}; if (r == "{" || r == "}") {r = "\\" r}; } else {p = ""} rest = rest " " p "{" r "}"; } ; if (l != "") {l = "\\nta{" l "}"};  printf "%s & %s & %s \\\\\n", l, m, rest}'

%% PROGRAM ::= STMT *
%% TY ::= ,int
%%      | ,obj
%% FORMAL ::= TY :id
%% FORMALS ::= %
%%           | FORMAL
%%           | FORMALSL
%% FORMALSL ::= FORMAL
%%            | FORMAL ,, FORMALSL
%% ACTUALS ::= %
%%           | EXPR
%%           | ACTUALSL
%% ACTUALSL ::= EXPR
%%            | EXPR ,, ACTUALSL
%% BLOCK ::= ,{ STMT * ,}
%% STMT ::= BLOCK
%%        | TY :id ,;
%%        | TY :id = EXPR ,;
%%        | EXPR ,;
%%        | EXPR ,:= EXPR ,;
%%        | ,;
%%        | ,if EXPR STMT ,;
%%        | ,if EXPR STMT ,else STMT
%%        | ,while ,( EXPR ,) STMT
%%        | ,break ,;
%%        | ,continue ,;
%%        | ,return ,;
%%        | ,return EXPR ,;
%%        | TY :id ,( FORMALSL ,) BLOCK
%%        | ,class :id ,( FORMALSL ,) BLOCK
%% EXPR ::= ,not EXPR
%%        | EXPR BINOP EXPR
%%        | EXPR ,is ,id
%%        | EXPR ,is TY
%%        | EXPR ,. EXPR
%%        | EXPR ,( ACTUALSL ,)
%%        | EXPR ,[ EXPR ,]
%%        | ,[ ACTUALSL ,]
%%        | ,[ ACTUALSL / EXPR ,]
%%        | :int
%%        | :string
%%        | ,NULL
%%        | :real
%%        | :id
%%        | ,( EXPR ,)
%% BINOP ::= ,==
%%         | ,!=
%%         | ,<
%%         | ,>
%%         | ,<=
%%         | ,>=
%%         | ,+
%%         | ,-
%%         | ,*
%%         | ,/

\begin{figure}
\[
\begin{array}{lcl}
  \nta{PROGRAM} & ::= &  \nt{STMT} {\star} \\
\nta{TY} & ::= &  \vterminal{int} \\
 & \VB &  \vterminal{obj} \\
\nta{FORMAL} & ::= &  \nt{TY}\  \terminal{id} \\
\nta{FORMALS} & ::= &  \varepsilon{} \\
 & \VB &  \nt{FORMAL} \\
 & \VB &  \nt{FORMALSL} \\
\nta{FORMALSL} & ::= &  \nt{FORMAL} \\
 & \VB &  \nt{FORMAL}\  \vterminal{,}\  \nt{FORMALSL} \\
\nta{ACTUALS} & ::= &  \varepsilon{} \\
 & \VB &  \nt{EXPR} \\
 & \VB &  \nt{ACTUALSL} \\
\nta{ACTUALSL} & ::= &  \nt{EXPR} \\
 & \VB &  \nt{EXPR}\  \vterminal{,}\  \nt{ACTUALSL} \\
\nta{BLOCK} & ::= &  \vterminal{\{}\  \nt{STMT} {\star}\  \vterminal{\}} \\
\nta{STMT} & ::= &  \nt{BLOCK} \\
 & \VB &  \nt{TY}\  \terminal{id}\  \vterminal{;} \\
 & \VB &  \nt{TY}\  \terminal{id}\  \nt{=}\  \nt{EXPR}\  \vterminal{;} \\
 & \VB &  \nt{EXPR}\  \vterminal{;} \\
 & \VB &  \nt{EXPR}\  \vterminal{:=}\  \nt{EXPR}\  \vterminal{;} \\
 & \VB &  \vterminal{;} \\
 & \VB &  \vterminal{if}\  \nt{EXPR}\  \nt{STMT}\  \vterminal{;} \\
 & \VB &  \vterminal{if}\  \nt{EXPR}\  \nt{STMT}\  \vterminal{else}\  \nt{STMT} \\
 & \VB &  \vterminal{while}\  \vterminal{(}\  \nt{EXPR}\  \vterminal{)}\  \nt{STMT} \\
 & \VB &  \vterminal{break}\  \vterminal{;} \\
 & \VB &  \vterminal{continue}\  \vterminal{;} \\
 & \VB &  \vterminal{return}\  \vterminal{;} \\
 & \VB &  \vterminal{return}\  \nt{EXPR}\  \vterminal{;} \\
 & \VB &  \nt{TY}\  \terminal{id}\  \vterminal{(}\  \nt{FORMALSL}\  \vterminal{)}\  \nt{BLOCK} \\
 & \VB &  \vterminal{class}\  \terminal{id}\  \vterminal{(}\  \nt{FORMALSL}\  \vterminal{)}\  \nt{BLOCK} \\
\nta{EXPR} & ::= &  \vterminal{not}\  \nt{EXPR} \\
 & \VB &  \nt{EXPR}\  \nt{BINOP}\  \nt{EXPR} \\
 & \VB &  \nt{EXPR}\  \vterminal{is}\  \vterminal{id} \\
 & \VB &  \nt{EXPR}\  \vterminal{is}\  \nt{TY} \\
 & \VB &  \nt{EXPR}\  \vterminal{.}\  \nt{EXPR} \\
 & \VB &  \nt{EXPR}\  \vterminal{(}\  \nt{ACTUALSL}\  \vterminal{)} \\
 & \VB &  \nt{EXPR}\  \vterminal{[}\  \nt{EXPR}\  \vterminal{]} \\
 & \VB &  \vterminal{[}\  \nt{ACTUALSL}\  \vterminal{]} \\
 & \VB &  \vterminal{[}\  \nt{ACTUALSL}\  \nt{/}\  \nt{EXPR}\  \vterminal{]} \\
 & \VB &  \terminal{int} \\
 & \VB &  \terminal{string} \\
 & \VB &  \vterminal{NULL} \\
 & \VB &  \terminal{id} \\
 & \VB &  \vterminal{(}\  \nt{EXPR}\  \vterminal{)} \\
\nta{BINOP} & ::= &  \vterminal{==} 
  \VB   \vterminal{!=} 
  \VB   \vterminal{<} 
  \VB   \vterminal{>} 
  \VB   \vterminal{<=} 
  \VB   \vterminal{>=} 
  \VB   \vterminal{+} 
  \VB   \vterminal{-} 
  \VB   \vterminal{*} 
  \VB   \vterminal{/} \\
\end{array}
\]
\caption{EBNF syntax for AttoL.  Here, $\nt{A}\star$ stands for zero or more repetitions of $\nt{A}$. }
\label{fig:syntax}
\end{figure}

\section{AttoVM: AttoL Compiler \& Runtime}

\subsection{Modules}\label{a:modules}

AttoVM consists of several modules, a few of which are generated indirectly by Python scripts.
Most of the modules use a header file of matching name (e.g., \texttt{ast.h} for \texttt{ast.c}).

\subsubsection{Frontend}

Several of the frontend modules are generated by the script \texttt{mk-parser.py}; these are marked by an asterisk (\textbf{*}) below.

\begin{tabular}{p{4cm}p{12cm}}
\texttt{lexer.l}\textbf{*} & The lexer is translated into a C program by another program, \texttt{flex}.
This code detects tokens and lexemes and reports them to the parser. \\
\texttt{parser.c}\textbf{*} & The parser identifies tokens and lexemes and attempts to construct an
abstract syntax tree from them. \\
\texttt{ast.c} & The abstract syntax tree (AST) represents the program after parsing and during program analysis.
This module also provides operations for creating and deallocating AST nodes.\\
\texttt{builtins.c} & This module contains built-in types and operations. \\
\texttt{class.c} & This module defines the notion of an AttoL `class'.
Boxed \texttt{int} values have a separate class defined therein. \\
\texttt{symbol-table.c} & The symbol table maps identifiers (internally stored as numbers) to \emph{symbol table entries}.
Symbol table entries store the bindings for each identifier, and are used throughout the system.  The
symbol table stores human-readable names, types, storage locations, function arities and other pertinent information.\\
\texttt{unparser.c}\textbf{*} & The \emph{unparser} contains code for printing out the abstract syntax tree.\\
\end{tabular}

\subsubsection{Analyses}

\texttt{analyses.h} captures all current program analyses.
They are part of the frontend and make intensive use of the symbol table.

\begin{tabular}{p{4cm}p{12cm}}
\texttt{name-analysis.c} & Name analysis identifies names in the input program and resolves names according to the scoping rules.  If name analysis succeeds, all names share appropriate symbol table entries.\\
\texttt{type-analysis.c} & Type analysis ensures that the program is correctly typed.  It adds implicit coercions (ad-hoc polymorphism) and
restructures some parts of the AST that relate to types. \\
\texttt{storage-allocation.c} & Storage allocation assigns storage (global, stack, or heap) to variables. \\
\end{tabular}

\subsection{Backend}

The assembler module (marked with an asterisk (\textbf{*})) is generated by the script \texttt{mk-codegen.py}.

\begin{tabular}{p{4cm}p{12cm}}
\texttt{assembler-buffer.c} & Assembler buffers are mechanisms for writing out assembly code.  They further define the concept of a jump label (\Cty{label\_t}), which can be resolved at a later time.
This module cooperates with \texttt{assembler.c} to define a disassembler. \\
\texttt{assembler.c}\textbf{*} & This module contains all supported assembly instructions. \\
\texttt{address-store.c} & The address store is used to support disassembly.  It is a hash table that records known function names and mentions them during disassembly, to increase readability. \\
\texttt{registers.c} & The register module defines the available registers.  \\
\texttt{baseline-backend.c} & The actual compiler, which takes the AST and emits assembly instructions to an assembly buffer. \\
\texttt{backend-test.c} & A test suite for the compiler. \\
\end{tabular}

\subsection{Run-time system}
\begin{tabular}{p{4cm}p{12cm}}
\texttt{runtime.c} & A generic run-time support module. \\
\texttt{object.c} & Object allocation. \\
\end{tabular}

\subsubsection{Support}

\begin{tabular}{p{4cm}p{12cm}}
\texttt{atl.c} & Main program. \\
\texttt{chash.c} & A generic hash table implementation for C. \\
\end{tabular}


\subsection{AttoVM: Abstract Syntax Tree}\label{a:ast}

The AttoVM AST consists of multiple different node types.
Each tree has two kinds of nodes: \emph{value nodes} and \emph{tree nodes}.
Value nodes contain literal values (64-bit integers, strings), while tree nodes
contain language constructions such as loops or blocks; consequently, tree nodes
may have child nodes.

Table~\ref{fig:valnodes} summarises the value nodes, and table~\ref{fig:rnodes} summarises the tree nodes.

The full current list is defined in \texttt{ast.h}.

When \emph{unparsing}, AttoVM prints out both kinds of nodes, together
with a number of \emph{attributes}.  Attributes are flags on AST nodes
that serve to reduce the number of AST nodes that we have to keep in memory.
Attributes are marked with a hash mark (\texttt{\#}); they are as follows:

\begin{itemize}
\item   \texttt{\#INT}: \Cty{int} type
\item   \texttt{\#OBJ}: \Cty{obj} type
\item   \texttt{\#LVALUE}: LValue (assignable memory reference)
\item   \texttt{\#DECL}: Identifier is declared or defined at this location
\item   \texttt{\#CONST}: Identifier marked as `constant'
\end{itemize}

All computations are represented via  \textsf{FUNAPP} nodes.  For example,
the AST below represents the program \texttt{print(1)} after name and type analysis:

\begin{verbatim}
  (FUNAPP#OBJ
    (SYM[-12] print):#OBJ
    (ACTUALS
      (FUNAPP#OBJ
        (SYM[-3] *convert):#OBJ
        (ACTUALS
          1:INT#INT
        )
      )
    )
  )
\end{verbatim}

As we can see, the number $1$ (type-attributed as \texttt{\#INT}) is first passed into function
\texttt{*convert} with symbol table entry $-3$.  \texttt{*convert} is an important function
that implements type coercions; while implicit in the source program, it becomes explicit \emph{in the AST}
during type analysis.
The expected result type is specified as an attribute in the function call:
\begin{verbatim}
        (SYM[-3] *convert):#OBJ
\end{verbatim}
Thus, coercion will produce an \texttt{\#OBJ}.
The reuslt of this function call is again a parameter to a function call,
namely to the built-in
\texttt{print} function:
\begin{verbatim}
    (SYM[-12] print):#OBJ
\end{verbatim}
This function prints its parameter and returns \Ckw{NULL}.

\begin{figure}[h]
\begin{tabular}{|p{3cm}|p{14cm}|}
\hline
\textbf{Node} & \textbf{Content} \\
\hline
\hline
\textsf{INT} &  64 bit signed integer number\\
\hline
\textsf{STRING} & character string \\
\hline
\textsf{NAME  } & name (converted into \textsf{ID} during name analysis) \\
\hline
\textsf{ID    } & identifier: a numeric index into  the symbol table.  Negative numbers are built-in identifiers, positive numbers are user-defined identifiers.  \\
\hline
\end{tabular}
\caption{Value nodes and their contents}\label{fig:valnodes}
\end{figure}

\begin{figure}[h]
\begin{tabular}{|p{2.8cm}|p{2cm}|p{11cm}|}
\hline
\textbf{Node} & \textbf{Children} & \textbf{Meaning} \\
\hline
\hline
\textsf{NULL}		& ---		& The constant value \Ckw{NULL}.\\
\hline
\textsf{ISPRIMTY}	& $(e)$		& Primitive type test; translated into \textsf{ISINSTANCE} or an \textsf{INT} constant during type analysis. \\
\hline
$\textsf{ISINSTANCE}^{\dagger}$	& $(e, \textit{id})$ & Type check: does $e$ have the type specified by \textit{id} at run-time? \\
\hline
\textsf{ARRAYLIST }	& $(e_0, \ldots, e_n)$ & Initialisation list for an array. \\
\hline
\textsf{ARRAYVAL}	& $(\textit{al}, e)$ & Allocate an array with initial values \textit{al} (always \textsf{ARRAYLIST}) of size $e$ (optional; if \Cpp{NULL}, we instead use the length of \textit{al}). \\
\hline
\textsf{ARRAYSUB}	& $(e_0, e_1)$	& Access to array entry $e_1$ in array $e_0$ ($\texttt{e}_0$[$\texttt{e}_1$]). \\
\hline
\textsf{ACTUALS}	& $(e_0, \ldots, e_n)$ & Actual parameter list for function/constructor/method call. \\
\hline
\textsf{FUNAPP}		& $(\textit{id}, \textit{a})$ & Executes the function identified by \textit{id} with \textsf{ACTUALS} list \textit{a}. \\
\hline
$\textsf{NEWINSTANCE}^\dagger$	&$(\textit{id}, \textit{a})$ & Constructor invocation, where \textit{id} is the class name and \textit{a} the argument list.\\
\hline
$\textsf{METHODAPP  }^\dagger$	&$(e, \textit{id}, \textit{a})$ & A method invocation of method \textit{id} on the object returned by $e$, with arguments (always \textsf{ACTUALS}) \textit{a} \\
\hline
\textsf{BLOCK}		& $(s_0, \ldots, s_n)$ & Execute all $s_i$ in sequence. \\
\hline
\textsf{SKIP} 		& ---		& No-op. \\
\hline
\textsf{IF}		& $(e, s_t, s_f)$ & Evaluates $e$ to an integer.  If non-zero, execute $s_t$, otherwise execute $s_f$.  $s_f$ may be \Cpp{NULL}, in which case it is a no-op. \\
\hline
\textsf{WHILE}		& $(e, s)$	& Evaluate $e$ for truth.  If non-zero, execute $s$.  Repeat ad inifintum. \\
\hline
\textsf{BREAK}		& ---		& \texttt{break} out of the most closely surrounding loop. \\
\hline
\textsf{CONTINUE} 	& ---		& \texttt{continue} over the rest of the loop body into the next loop check/iteration. \\
\hline
\textsf{RETURN     }	&$(e^?) $ & Return; may optionally have a return value expression $e$ as a child. \\
\hline
\textsf{VARDECL}	& $(\textit{id}, e)$ & Declaration of a variable \textit{id}.  Child node $e$ is the optional initialiser value, or \Cpp{NULL} otherwise. \\
\hline
\textsf{ASSIGN}		& $(e_0, e_1)$  & Assigns  $e_1$ to $e_0$. $e_0$ must be an LValue (see below). \\
\hline
\textsf{MEMBER}		& $(e, \textit{id}) $ & Field access to field \textit{id} on the object computed by $e$. \\ 
\hline
\textsf{FORMALS    }	&$(\textit{id}_0, \ldots, \textit{id}_n)$ & A sequence of identifiers with type attributes; used as a list of formal parameters.\\
\hline
\textsf{FUNDEF     }	&$(\textit{id}, \textit{f}, s)$ & Function definition for function \textit{id} with parameters \textit{f} (always \textsf{FORMALS}) and body $s$ (always a \textsf{BLOCK})\\
\hline
\textsf{CLASSDEF   }	&$(\textit{id}, \textit{f}, s, c)$ & Class definition with name \textit{id} and constructor arguments \textit {f} (\textsf{FORMALS}).  The class body $s$ is a \textsf{BLOCK}.  $c$ is initially \Cpp{NULL}, but during type analysis, AttoVM builds a constructor function (\textsf{FUNDEF}) that is placed in $c$.\\
\hline
\end{tabular}
\caption{Tree nodes with their contents.
  Expressions/computations are generally expressed as \textsf{FUNAPP}s.  LValues are
  an \textsf{ID} from Figure~\ref{fig:valnodes}, \textsf{ARRAYSUB}, or \textsf{MEMBER}.  AST nodes marked with ($\dagger$)
  are not generated by the parser, but instead during type analysis.
}\label{fig:rnodes}
\end{figure}


\subsection{Emitting 2OPM Code}\label{a:2omp}

\subsubsection{Generating 2OPM assembly with \texttt{assembler.c}}

The \texttt{assembler.h}/\texttt{assembler.c} library provides
facilities for emitting assembly code.  This assembly code must be
written to a \emph{machine code buffer}, type \Cty{buffer\_t}, from
\texttt{assembler-buffer.h}.  Such a buffer is allocated in a memory region specifically
allocated to allow reading, writing, and execution.

Once you have access to such a buffer, you can e.g. write the following to increment
 register  \texttt{\$v0} by $3$:

\texttt{emit\_addi(buf, REGISTER\_T0, 3);}

To load a known pointer \texttt{addr} into  register \texttt{\$t1}, you can write:

\texttt{emit\_la(buf, REGISTER\_T1, addr);}

The exact list is defined in \texttt{assembler.h}, with the prototypes
matching the 2OPM manual.

\subsubsection{Jump Labels}
Several of the operations in \texttt{assembler.h} require jump labels.
In some cases (jumping backwards) you know the address you want to
jump to, but if you want to jump to a location that you haven't generated yet,
you don't yet know what its exact address will be.  To address this challenge,
AttoVM uses a special type \Cty{label\_t} to represent locations in which jump labels will be added later.

To utilise such a label, first invoke the branch function you want to use, passing a pointer to
storage that you have allocated for the label (typically a stack-dynamic variable).  For example:

\vspace{0.5cm}
\begin{tabular}{ll}
&\texttt{\Cty{label\_t} label;}\\
$A$&\texttt{emit\_beqz(buf, REGISTER\_T0, \&label);}\\
\end{tabular}
\vspace{0.5cm}

At a later point in time, you can (and must!) update this label, using
the function \texttt{buffer\_setlabel2(\&label, buf)}.  Continuing our example:

\vspace{0.5cm}

\begin{tabular}{ll}
&\texttt{emit\_addi(buf, REGISTER\_T0, 3);} \\
$B$&\texttt{buffer\_setlabel2(\&label, buf);} \\
&\texttt{emit\_la(buf, REGISTER\_T1, addr);} \\
\end{tabular}

\vspace{0.5cm}

In this example you are telling the backend that you want to
jump from $A$ to $B$, if the condition ($\texttt{\$t0} = 0$) is satisfied in $A$.

To jump backwards, you can use \texttt{buffer\_target()} and \texttt{buffer\_setlabel()} instead:

\vspace{0.5cm}

\begin{tabular}{ll}
  &\texttt{emit\_addi(buf, REGISTER\_T0, 3);} \\
  $B$&\texttt{\Cty{void *}addr = buffer\_target(buf);} \\
  $C$ &\texttt{emit\_li(buf, REGISTER\_T1, 23);} \\
  & \ldots \\
  &\texttt{\Cty{label\_t} label;}\\
  $A$&\texttt{emit\_beqz(buf, REGISTER\_T0, \&label);}\\
  & \texttt{buffer\_setlabel(\&label, addr);}\\
\end{tabular}

\vspace{0.5cm}

In this example, we will jump from $A$ to $C$, if the condition is satisfied.
Here, $B$ extracts the memory address of the instruction we will want to jump back to (i.e.,
the instruction we're emitting in $C$).  We later reference this address when calling \texttt{buffer\_setlabel()}.


\subsubsection{Registers}

For convenience, \texttt{registers.h} defines the names of the following registers:

\vspace{0.5cm}

\begin{tabular}{|l|l|}
\hline
\textbf{Register} & \textbf{Preprocessor macro with register number} \\
\hline
  \texttt{\$t0} & \texttt{REGISTER\_T0} \\
\hline
  \texttt{\$t1} & \texttt{REGISTER\_T1} \\
\hline
  \texttt{\$v0} & \texttt{REGISTER\_V0} \\
\hline
  \texttt{\$gp} & \texttt{REGISTER\_GP} (static memory) \\
\hline
\texttt{\$sp} & \texttt{REGISTER\_SP}  \\
\hline
\texttt{\$fp} & \texttt{REGISTER\_FP}  \\
\hline
\end{tabular}

\begin{figure}
\begin{tabular}{|l|l|l|l|l|l|}
\hline
\textbf{Return} & \textbf{Temporary} & \textbf{Saved} & \textbf{Parameter} & \textbf{Special} & \textbf{Program counter}\\
\hline
\hline
\texttt{\$v0} & \texttt{\$t0} & \texttt{\$s0} & \texttt{\$a0} & \texttt{\$gp} & \texttt{\$pc} \\
              & \texttt{\$t1} & \texttt{\$s1} & \texttt{\$a1} & \texttt{\$sp} &               \\
              &               & \texttt{\$s2} & \texttt{\$a2} & \texttt{\$fp} &               \\
              &               & \texttt{\$s3} & \texttt{\$a3} &               &               \\
              &               &               & \texttt{\$a4} &               &               \\
              &               &               & \texttt{\$a5} &               &               \\
\hline
\end{tabular}
\caption{2OPM registers, overview}
\end{figure}


\end{document}


